\pagebreak

\sect{Plato's Euthyphro}

\speaker{Εὐθύφρων}
\versification{[2a]}
τί
νεώτερον,
ὦ
Σώκρατες,
γέγονεν,
ὅτι
σὺ
τὰς
ἐν
Λυκείῳ
καταλιπὼν
διατριβὰς\footnote{διατριβή - a way of spending time}
ἐνθάδε\footnote{ἐνθάδε - here}
νῦν
διατρίβεις
περὶ
τὴν
τοῦ
βασιλέως
στοάν;
οὐ
γάρ
που\footnote{"For certainly there is no...is there?" See discussion in Smyth 2651 and FCC on this.}
καὶ
σοί
γε
δίκη
τις
οὖσα
τυγχάνει
πρὸς
τὸν
βασιλέα
ὥσπερ
ἐμοί.\footnote{Dative of the possessor. See Smyth 1476ff.}

\speaker{Σωκράτης}
οὔτοι
δὴ
Ἀθηναῖοί
γε,
ὦ
Εὐθύφρων,
δίκην
αὐτὴν
καλοῦσιν
ἀλλὰ
γραφήν.\footnote{Moving from generic to specific, δίκη (more general word for case) to a γραφή (a case about actions against the state)}

\speaker{Εὐθύφρων}
\versification{[2b]}
τί
φῄς;
γραφὴν
σέ
τις,
ὡς
ἔοικε,
γέγραπται·
οὐ
γὰρ
ἐκεῖνό
γε
καταγνώσομαι,
ὡς
σὺ
ἕτερον.

\speaker{Σωκράτης}
οὐ
γὰρ
οὖν.

\speaker{Εὐθύφρων}
ἀλλὰ
σὲ
ἄλλος;

\speaker{Σωκράτης}
πάνυ
γε.

\speaker{Εὐθύφρων}
τίς
οὗτος;

\speaker{Σωκράτης}
οὐδ᾽
αὐτὸς
πάνυ
τι
γιγνώσκω,
ὦ
Εὐθύφρων,
τὸν
ἄνδρα,
νέος
γάρ
τίς
μοι
φαίνεται
καὶ
ἀγνώς·
ὀνομάζουσι
μέντοι
αὐτόν,
ὡς
ἐγᾦμαι,\footnote{ἐγώ + οἶμαι. οἶμαι - I suppose, expect, imagine.}
Μέλητον.
ἔστι
δὲ
τῶν
δήμων\footnote{δῆμος - people}
Πιτθεύς,
εἴ
τινα
νῷ\footnote{νόος - mind}
ἔχεις
Πιτθέα
Μέλητον
οἷον
τετανότριχα\footnote{τετανότριξ - having long hair}
καὶ
οὐ
πάνυ
εὐγένειον,\footnote{εὐγένειος - well-maned (referring here to his beard?)}
ἐπίγρυπον\footnote{ἐπίγρυπος - somewhat-hooked (nose)}
δέ.

\speaker{Εὐθύφρων}
οὐκ
ἐννοῶ,\footnote{ἐννοέω - I think, consider}
ὦ
Σώκρατες·
ἀλλὰ
δὴ
τίνα
γραφήν
\versification{[2c]}
σε
γέγραπται;

\speaker{Σωκράτης}
ἥντινα;
οὐκ
ἀγεννῆ,\footnote{ἀγεννή - minor}
ἔμοιγε\footnote{dat. sg. of ἐγώ}
δοκεῖ·
τὸ
γὰρ
νέον
ὄντα
τοσοῦτον
πρᾶγμα\footnote{probably "lawsuit" here.}
ἐγνωκέναι
οὐ
φαῦλόν\footnote{φαῦλος - small, paltry}
ἐστιν.
ἐκεῖνος
γάρ,
ὥς
φησιν,
οἶδε
τίνα
τρόπον
οἱ
νέοι
διαφθείρονται
καὶ
τίνες
οἱ
διαφθείροντες
αὐτούς.
καὶ
κινδυνεύει\footnote{"It might probably be" here, a seemingly less-common usage.}
σοφός
τις
εἶναι,
καὶ
τὴν
ἐμὴν
ἀμαθίαν\footnote{ἀμαθία - ignorance}
κατιδὼν\footnote{κατεῖδον - I look down; observe}
ὡς
διαφθείροντος
τοὺς
ἡλικιώτας\footnote{ἡλικιίωτης - an equal in age, fellow, comrade}
αὐτοῦ,
ἔρχεται
κατηγορήσων
μου
ὥσπερ
πρὸς
μητέρα\footnote{As though the city were his mother}
πρὸς
τὴν
πόλιν.
καὶ
φαίνεταί
μοι
τῶν
πολιτικῶν\footnote{πολιτικός - of or relating to citizens; public; official; statesman; belonging to the state; common, ordinary}
\versification{[2d]}
μόνος
ἄρχεσθαι
ὀρθῶς·
ὀρθῶς
γάρ
ἐστι
τῶν
νέων
πρῶτον
ἐπιμεληθῆναι\footnote{ἐπιμελέομαι - I take care of, have charge of; have the management of}
ὅπως
ἔσονται
ὅτι
ἄριστοι,
ὥσπερ
γεωργὸν
ἀγαθὸν
τῶν
νέων
φυτῶν\footnote{φυτόν - a plant, tree}
εἰκὸς
πρῶτον
ἐπιμεληθῆναι,
μετὰ
δὲ
τοῦτο
καὶ
τῶν
ἄλλων.
καὶ
δὴ
καὶ
Μέλητος
ἴσως\footnote{ἴσως - equally; perhaps; it may be that}
πρῶτον
\versification{[3a]}
μὲν
ἡμᾶς
ἐκκαθαίρει\footnote{ἐκκαθαίρω - I clean out, clean thoroughly}
τοὺς
τῶν
νέων
τὰς
βλάστας\footnote{βλάστη - shoot}
διαφθείροντας,
ὥς
φησιν·
ἔπειτα
μετὰ
τοῦτο
δῆλον
ὅτι
τῶν
πρεσβυτέρων
ἐπιμεληθεὶς
πλείστων
καὶ
μεγίστων
ἀγαθῶν
αἴτιος
τῇ
πόλει
γενήσεται,
ὥς
γε
τὸ
εἰκὸς
συμβῆναι\footnote{συμβαίνω - come together; come to pass}
ἐκ
τοιαύτης
ἀρχῆς
ἀρξαμένῳ. 




%========================================== SECTION 3 ==========================================%



\speaker{Εὐθύφρων}
βουλοίμην
ἄν,
ὦ
Σώκρατες,
ἀλλ᾽
ὀρρωδῶ\footnote{ὀρρωδέω - I fear, dread, shrink from}
μὴ
τοὐναντίον\footnote{"the opposite"}
γένηται·
ἀτεχνῶς\footnote{ἀτεχνῶς - simply}
γάρ
μοι
δοκεῖ
ἀφ᾽
ἑστίας\footnote{ἑστία - the hearth, fireside. Idiomatically, "at its very heart". Meletos is beginning at the heart of the city to do wrong.}
ἄρχεσθαι
κακουργεῖν\footnote{κακουργέω - I do evil, work wickedness}
τὴν
πόλιν,
ἐπιχειρῶν
ἀδικεῖν
σέ.
καί
μοι
λέγε,
τί
καὶ
ποιοῦντά
σέ
φησι
διαφθείρειν
τοὺς
νέους; 

\speaker{Σωκράτης}
\versification{[3b]}
ἄτοπα,\footnote{ἄτοπος - improper, unrighteous, perverse}
ὦ
θαυμάσιε,\footnote{θαυμάσιος - wonderful, admirable}
ὡς
οὕτω
γ᾽
ἀκοῦσαι.\footnote{"to hear it just like that"}
φησὶ
γάρ
με
ποιητὴν
εἶναι
θεῶν,
καὶ
ὡς
καινοὺς
ποιοῦντα
θεοὺς
τοὺς
δ᾽
ἀρχαίους
οὐ
νομίζοντα
ἐγράψατο
τούτων
αὐτῶν
ἕνεκα,
ὥς
φησιν.

\speaker{Εὐθύφρων}
μανθάνω,
ὦ
Σώκρατες·
ὅτι
δὴ
σὺ
τὸ
δαιμόνιον
φῂς
σαυτῷ
ἑκάστοτε\footnote{ἑκάστοτε - each time, always}
γίγνεσθαι.
ὡς
οὖν
καινοτομοῦντός\footnote{καινοτομέω - I cut fresh into; innovate}
σου
περὶ
τὰ
θεῖα
γέγραπται
ταύτην
τὴν
γραφήν,
καὶ
ὡς
διαβαλῶν\footnote{διαβάλλω - I slander}
δὴ
ἔρχεται
εἰς
τὸ
δικαστήριον,\footnote{δικαστήριον - a court of justice}
εἰδὼς
ὅτι
εὐδιάβολα\footnote{εὐδιάβολος - easy to misrepresent}
τὰ
τοιαῦτα
πρὸς
τοὺς
πολλούς.
καὶ
ἐμοῦ
γάρ
τοι,
\versification{[3c]}
ὅταν
τι
λέγω
ἐν
τῇ
ἐκκλησίᾳ
περὶ
τῶν
θείων,
προλέγων
αὐτοῖς
τὰ
μέλλοντα,
καταγελῶσιν\footnote{\parse{pr act ind 3rd pl} καταγελάω - I deride, laugh scornfully at.}
ὡς
μαινομένου·
καίτοι
οὐδὲν
ὅτι
οὐκ
ἀληθὲς
εἴρηκα
ὧν
προεῖπον,
ἀλλ᾽
ὅμως
φθονοῦσιν\footnote{φθονέω - I envy}
ἡμῖν
πᾶσι
τοῖς
τοιούτοις.
ἀλλ᾽
οὐδὲν
αὐτῶν
χρὴ
φροντίζειν,
ἀλλ᾽
ὁμόσε\footnote{ὁμόσε - to one and the same place}
ἰέναι.\footnote{\parse{pres act inf} of εἶμι, I go. "Close with the enemy" according to Emlyn-Jones.}

\speaker{Σωκράτης}
ὦ
φίλε
Εὐθύφρων,
ἀλλὰ
τὸ
μὲν
καταγελασθῆναι
ἴσως
οὐδὲν
πρᾶγμα.\footnote{"Is of no consequence", Emlyn-Jones}
Ἀθηναίοις
γάρ
τοι,
ὡς
ἐμοὶ
δοκεῖ,
οὐ
σφόδρα
μέλει\footnote{μέλω - I care for; pass: to be an object of care}
ἄν
τινα
δεινὸν
οἴωνται\footnote{οἴομαι - I suppose, expect, imagine}
εἶναι,
μὴ
μέντοι
διδασκαλικὸν\footnote{διδασκαλικός - fit for teaching, instructive}
τῆς
αὑτοῦ
σοφίας·
ὃν
δ᾽
ἂν
καὶ
ἄλλους
οἴωνται
\versification{[3d]}
ποιεῖν
τοιούτους,
θυμοῦνται,\footnote{θυμόω - I am very angry}
εἴτ᾽
οὖν
φθόνῳ
ὡς
σὺ
λέγεις,
εἴτε
δι᾽
ἄλλο
τι.

\speaker{Εὐθύφρων}
τούτου
οὖν
πέρι
ὅπως
ποτὲ
πρὸς
ἐμὲ
ἔχουσιν,
οὐ
πάνυ
ἐπιθυμῶ\footnote{ἐπιθυμέω - I desire}
πειραθῆναι.\footnote{πειράω - I try, attempt}

\speaker{Σωκράτης}
ἴσως
γὰρ
σὺ
μὲν
δοκεῖς
σπάνιον\footnote{σπάνιος - rare, scarce, scanty}
σεαυτὸν
παρέχειν
καὶ
διδάσκειν
οὐκ
ἐθέλειν
τὴν
σεαυτοῦ
σοφίαν·
ἐγὼ
δὲ
φοβοῦμαι
μὴ
ὑπὸ
φιλανθρωπίας\footnote{φιλανθρωπία - benevolence}
δοκῶ
αὐτοῖς
ὅτιπερ
ἔχω
ἐκκεχυμένως\footnote{ἐκκεχυμένως - profusely}
παντὶ
ἀνδρὶ
λέγειν,
οὐ
μόνον
ἄνευ
μισθοῦ,
ἀλλὰ
καὶ
προστιθεὶς
ἂν
ἡδέως\footnote{ἡδέως - gladly, with pleasure}
εἴ
τίς
μου
ἐθέλει
ἀκούειν.
εἰ
μὲν
οὖν,
ὃ
νυνδὴ
ἔλεγον,
μέλλοιέν\footnote{\parse{pr act opt 3rd pl} of μέλλω}
μου
καταγελᾶν
ὥσπερ
\versification{[3e]}
σὺ
φῂς
σαυτοῦ,
οὐδὲν
ἂν
εἴη
ἀηδὲς\footnote{ἀηδής - unpleasant to the taste, distasteful}
παίζοντας\footnote{παίζω - I play}
καὶ
γελῶντας
ἐν
τῷ
δικαστηρίῳ
διαγαγεῖν·\footnote{διάγω - I pass time; carry over}
εἰ
δὲ
σπουδάσονται,
τοῦτ᾽
ἤδη
ὅπῃ\footnote{ὅπη - by which way}
ἀποβήσεται\footnote{ἀποβαίνω - (of events) to happen, turn out}
ἄδηλον\footnote{ἄδηλος - unseen, inconspicuous}
πλὴν
ὑμῖν
τοῖς
μάντεσιν.\footnote{μάντις - one who divines, seer}

\speaker{Εὐθύφρων}
ἀλλ᾽
ἴσως\footnote{ἵσως - perhaps, probably}
οὐδὲν
ἔσται,
ὦ
Σώκρατες,
πρᾶγμα,
ἀλλὰ
σύ
τε
κατὰ
νοῦν\footnote{"satisfactorily", Burnet}
ἀγωνιῇ\footnote{\parse{fut m ind 2d sg} of ἀγωνίζομαι}
τὴν
δίκην,
οἶμαι
δὲ
καὶ
ἐμὲ
τὴν
ἐμήν.

\speaker{Σωκράτης}
ἔστιν
δὲ
δὴ
σοί,
ὦ
Εὐθύφρων,
τίς
ἡ
δίκη;
φεύγεις\footnote{In the context of legal matters, φεύγω takes on the idea of defending.}
αὐτὴν
ἢ
διώκεις;

\speaker{Εὐθύφρων}
διώκω.

\speaker{Σωκράτης}
τίνα; 



%========================================== SECTION 4 ==========================================%



\speaker{Εὐθύφρων}
\versification{[4a]}
ὃν
διώκων
αὖ\footnote{"again", pointing back to his earlier negative court experiences.}
δοκῶ\footnote{The people of the court consider Euthyphro μαίνεσθαι.}
μαίνεσθαι.\footnote{In this case "crazy", not "angry".}

\speaker{Σωκράτης}
τί
δέ;
πετόμενόν
τινα
διώκεις;\footnote{proverbial "go on a goose chase"}

\speaker{Εὐθύφρων}
πολλοῦ\footnote{modifying δεῖ, saying that much would be necessary for the person to πέτεσθαι, given his age.}
γε
δεῖ
πέτεσθαι,
ὅς
γε
τυγχάνει
ὢν
εὖ
μάλα\footnote{μάλα - very}
πρεσβύτης.

\speaker{Σωκράτης}
τίς
οὗτος;

\speaker{Εὐθύφρων}
ὁ
ἐμὸς
πατήρ.

\speaker{Σωκράτης}
ὁ
σός,
ὦ
βέλτιστε;\footnote{βέλτιστος - dear friend}

\speaker{Εὐθύφρων}
πάνυ\footnote{πάνυ - altogether, entirely}
μὲν
οὖν.

\speaker{Σωκράτης}
ἔστιν
δὲ
τί
τὸ
ἔγκλημα\footnote{ἔγκλημα - accusation, charge}
καὶ
τίνος
ἡ
δίκη;

\speaker{Εὐθύφρων}
φόνου,
ὦ
Σώκρατες.

\speaker{Σωκράτης}
Ἡράκλεις.
ἦ\footnote{ἦ - truly}
που,\footnote{πού - anywhere, somewhere}
ὦ
Εὐθύφρων,
ἀγνοεῖται
ὑπὸ
τῶν
πολλῶν
ὅπῃ\footnote{ὅπη - by which. ὅπῃ...ἔχει as a whole is the subject of ἀγνοεῖται.}
ποτὲ
ὀρθῶς
ἔχει·\footnote{Burnet mentions that emendations are suggested for this text (which he thinks are unnecessary), so at least others find this confusing as well.}
οὐ
γὰρ
οἶμαί
γε
τοῦ
ἐπιτυχόντος\footnote{ἐπιτυγχάνω - I obtain, attain to, reach; find, happen upon. LSJ lists this passage with the latter meaning but substantivally, "the first person happened upon" or "any chance person", with Burnet and Emlyn-Jones. In other words, most people. Perhaps contra Bailly, whose explanation I don't understand yet.}
\versification{[4b]}
[ὀρθῶς]
αὐτὸ
πρᾶξαι
ἀλλὰ
πόρρω\footnote{πόρρω + genitive = "far into", Emlyn-Jones. LSJ lists this under the idea of reaching a high point of something, in this case wisdom.}
που
ἤδη
σοφίας
ἐλαύνοντος.

\speaker{Εὐθύφρων}
πόρρω
μέντοι
νὴ\footnote{"by", seems to be used when addressing deities.}
Δία,
ὦ
Σώκρατες.

\speaker{Σωκράτης}
ἔστιν
δὲ
δὴ
τῶν
οἰκείων
τις
ὁ
τεθνεὼς
ὑπὸ
τοῦ
σοῦ
πατρός;
ἢ
δῆλα
δή;
οὐ\footnote{Comparable to the massing of particles in 2a3.}
γὰρ
ἄν
που
ὑπέρ
γε
ἀλλοτρίου\footnote{ἀλλότριος - belonging to another. In this case, referring to someone outside the household.}
ἐπεξῇσθα\footnote{ἐπέξειμι - I go out against; prosecute. The one prosecuted in the dative, the charge in the genitive.}
φόνου
αὐτῷ.

\speaker{Εὐθύφρων}
γελοῖον,
ὦ
Σώκρατες,
ὅτι
οἴει
τι
διαφέρειν
εἴτε
ἀλλότριος
εἴτε
οἰκεῖος
ὁ
τεθνεώς,
ἀλλ᾽
οὐ
τοῦτο
μόνον
δεῖν
φυλάττειν,
εἴτε
ἐν
δίκῃ\footnote{ἐν δίκῃ - lawfully}
ἔκτεινεν
ὁ
κτείνας
εἴτε
μή,
καὶ
εἰ
μὲν
ἐν
δίκῃ,
ἐᾶν,\footnote{ἐάω - I permit}
εἰ
δὲ
μή,
ἐπεξιέναι,
ἐάνπερ
ὁ
κτείνας
συνέστιός\footnote{συνέστιος - sharing one's hearth or house}
\versification{[4c]}
σοι
καὶ
ὁμοτράπεζος\footnote{ὁμοτράπεζος - sharing a table}
ᾖ·
ἴσον
γὰρ
τὸ
μίασμα\footnote{μίασμα - stain, defilement}
γίγνεται
ἐὰν
συνῇς
τῷ
τοιούτῳ
συνειδὼς
καὶ
μὴ
ἀφοσιοῖς\footnote{ἀφοσιόω - to purify from guilt}
σεαυτόν
τε
καὶ
ἐκεῖνον
τῇ
δίκῃ
ἐπεξιών.
ἐπεὶ
ὅ
γε
ἀποθανὼν
πελάτης\footnote{πελάτης - one who seeks protection, client, dependant}
τις
ἦν
ἐμός,\footnote{He was a laborer for Euthyphro}
καὶ
ὡς
ἐγεωργοῦμεν
ἐν
τῇ
Νάξῳ,\footnote{A city under the control of Athens until 404.}
ἐθήτευεν\footnote{θητεύω - I am a serf, menial laborer}
ἐκεῖ
παρ᾽
ἡμῖν.
παροινήσας\footnote{παροινέω - I play drunken tricks. Michael, we've seen this word before. Check out True Story, 107-8 and story of Diogenes.}
οὖν
καὶ
ὀργισθεὶς
τῶν
οἰκετῶν
τινι
τῶν
ἡμετέρων
ἀποσφάττει\footnote{ἀποσφάζω - I the throat of x}
αὐτόν.
ὁ
οὖν
πατὴρ
συνδήσας
τοὺς
πόδας
καὶ
τὰς
χεῖρας
αὐτοῦ,
καταβαλὼν
εἰς
τάφρον\footnote{τάφρος - ditch, tent}
τινά,
πέμπει
δεῦρο
ἄνδρα
πευσόμενον\footnote{πυνθάνομαι - I learn by hearsay, inquiry}
τοῦ
ἐξηγητοῦ\footnote{ἐξηγητής - one who leads on, adviser}
ὅτι
χρείη\footnote{\parse{pr act opt 3rd sg} of χρή - it is necessary}
\versification{[4d]}
ποιεῖν.
ἐν
δὲ
τούτῳ
τῷ
χρόνῳ
τοῦ
δεδεμένου
ὠλιγώρει\footnote{ὀλιγωρέω - I despise, hold in low esteem}
τε
καὶ
ἠμέλει\footnote{ἀμελέω - I neglect, disregard}
ὡς
ἀνδροφόνου
καὶ
οὐδὲν
ὂν\footnote{\parse{pr act ptcpl} of εἰμί}
πρᾶγμα
εἰ
καὶ
ἀποθάνοι,\footnote{\parse{aor act opt 3rd sg} of ἀποθνήσκω}
ὅπερ\footnote{ὅπερ - the very one which}
οὖν
καὶ
ἔπαθεν·
ὑπὸ
γὰρ
λιμοῦ\footnote{λιμός - hunger}
καὶ
ῥίγους\footnote{ῥῖγος - frost, cold}
καὶ
τῶν
δεσμῶν
ἀποθνῄσκει
πρὶν
τὸν
ἄγγελον
παρὰ
τοῦ
ἐξηγητοῦ
ἀφικέσθαι.
ταῦτα
δὴ
οὖν
καὶ
ἀγανακτεῖ\footnote{ἀγανακτέω - I am indignant}
ὅ
τε
πατὴρ
καὶ
οἱ
ἄλλοι
οἰκεῖοι,
ὅτι
ἐγὼ
ὑπὲρ
τοῦ
ἀνδροφόνου
τῷ
πατρὶ
φόνου
ἐπεξέρχομαι
οὔτε
ἀποκτείναντι,
ὥς
φασιν
ἐκεῖνοι,
οὔτ᾽
εἰ
ὅτι
μάλιστα
ἀπέκτεινεν,
ἀνδροφόνου
γε
ὄντος
τοῦ
ἀποθανόντος,
οὐ
δεῖν
φροντίζειν
ὑπὲρ
τοῦ
τοιούτου
--
ἀνόσιον\footnote{ἀνόσιος - unholy, profane}
\versification{[4e]}
γὰρ
εἶναι
τὸ
ὑὸν\footnote{ὑόν = υἱόν}
πατρὶ
φόνου
ἐπεξιέναι\footnote{ἐπέξειμι - I prosecute. Can take a dative or accusative direct object.}
-
κακῶς
εἰδότες,
ὦ
Σώκρατες,
τὸ
θεῖον
ὡς
ἔχει
τοῦ
ὁσίου\footnote{ὅσιος - holy}
τε
πέρι
καὶ
τοῦ
ἀνοσίου.

\speaker{Σωκράτης}
σὺ
δὲ
δὴ
πρὸς
Διός,
ὦ
Εὐθύφρων,
οὑτωσὶ\footnote{οὑτωσί - strengthened form of οὕτως}
ἀκριβῶς\footnote{ἀκριβῶς - carefully, precisely}
οἴει
ἐπίστασθαι\footnote{Probably from ἐπίσταμαι and not ἐφίστημι.}
περὶ
τῶν
θείων
ὅπῃ
ἔχει,
καὶ
τῶν
ὁσίων
τε
καὶ
ἀνοσίων,
ὥστε
τούτων
οὕτω
πραχθέντων
ὡς
σὺ
λέγεις,
οὐ
φοβῇ
δικαζόμενος
τῷ
πατρὶ
ὅπως
μὴ
αὖ\footnote{Pointing back to the deed of his father. Socrates is questioning if he is inadvertently doing wrong as he supposes his father is.}
σὺ
ἀνόσιον
πρᾶγμα
τυγχάνῃς
πράττων;

\speaker{Εὐθύφρων}
οὐδὲν
γὰρ
ἄν
μου
ὄφελος
εἴη,
ὦ
Σώκρατες,
οὐδέ
\versification{[5a]}
τῳ\footnote{TODO: Update. This is unclear to me but probably a form of τοί (an enclytic, which explains the lack of accent), meaning something like "therefore". See Smyth section 2987. See also the short entry in LSJ on τῷ for more relevant ideas.}
ἂν
διαφέροι\footnote{\parse{pr act opt 3rd sg} of διαφέρω.}
Εὐθύφρων
τῶν
πολλῶν
ἀνθρώπων,
εἰ
μὴ
τὰ
τοιαῦτα
πάντα
ἀκριβῶς
εἰδείην.\footnote{\parse{perf act opt ,1st sg} of οἶδα.}



%========================================== SECTION 5 ==========================================%



\speaker{Σωκράτης}
ἆρ᾽
οὖν
μοι,
ὦ
θαυμάσιε
Εὐθύφρων,
κράτιστόν
ἐστι
μαθητῇ
σῷ
γενέσθαι,\footnote{Oddly, this stative verb joins to datives, μοι and μαθητῇ σῷ. Bailly calls this complimentary but you could probably fit this in with Wallace's category of epexegetical infinitive, modifying κράτιστον.}
καὶ
πρὸ
τῆς
γραφῆς
τῆς
πρὸς
Μέλητον
αὐτὰ
ταῦτα
προκαλεῖσθαι\footnote{προκαλέω - I challenge. In Athenian law this is a technical term, referring to a πρόκλησις, a pre-trial event during which one party either asks for something or demands something from the other party. Whatever came out of this could be used in the trial (Burnet, 108; Bailly, 43)}
αὐτόν,
λέγοντα
ὅτι
ἔγωγε
καὶ
ἐν
τῷ
ἔμπροσθεν
χρόνῳ
τὰ
θεῖα
περὶ
πολλοῦ
ἐποιούμην
εἰδέναι,\footnote{ἐποιούμην εἰδέναι - combined express some notion of endeavoring to know.}
καὶ
νῦν
ἐπειδή
με\footnote{subject of the infinitive ἐξαμαρτάνειν in indirect discourse with φησι.}
ἐκεῖνος
αὐτοσχεδιάζοντά\footnote{αὐτοσχεδιάζω - I speak off-hand, extemporize; speak unadvisedly. LSJ lists the latter as the meaning here.}
φησι
καὶ
καινοτομοῦντα\footnote{καινοτομέω - I innovate}
περὶ
τῶν
θείων
ἐξαμαρτάνειν,\footnote{ἐξαμαρτάνω - I fail}
μαθητὴς
δὴ
γέγονα
σός
--
“καὶ
εἰ
μέν,
ὦ
Μέλητε”,
φαίην\footnote{\parse{pr act opt 1st sg} of φήμι}
ἄν,
“Εὐθύφρονα
ὁμολογεῖς
\versification{[5b]}
σοφὸν
εἶναι
τὰ
τοιαῦτα,
[καὶ]
ὀρθῶς
νομίζειν\footnote{Fowler applies this to Socrates, Jowett to Euthyphro. }
καὶ
ἐμὲ
ἡγοῦ\footnote{TODO: Imperitive follow me}
καὶ
μὴ
δικάζου·\footnote{"do (not) prosecute me"}
εἰ
δὲ
μή,
ἐκείνῳ
τῷ
διδασκάλῳ
λάχε\footnote{\parse{aor act impv 2nd sg}.}
δίκην\footnote{λαγχάνω + δίκην means "allow to bring a suit" in Athenian law terminology. Cf. LSJ λαγχάνω I.3}
πρότερον
ἢ
ἐμοί,
ὡς
τοὺς
πρεσβυτέρους
διαφθείροντι\footnote{Dat. in agreement with the previous διδασκάλῳ.}
ἐμέ
τε
καὶ
τὸν
αὑτοῦ
πατέρα,
ἐμὲ
μὲν
διδάσκοντι,
ἐκεῖνον
δὲ
νουθετοῦντί\footnote{νουθετέω - exhort, correct; warn; advise}
τε
καὶ
κολάζοντι”\footnote{κολάζω - chastise, correct, punish; check, restrain}
--
καὶ
ἂν\footnote{TODO: shortened form of ἐάν}
μή
μοι
πείθηται\footnote{πείθω - persuade}
μηδὲ
ἀφίῃ\footnote{\parse{pres subj act 3rd sg} of ἀφίημι.}
τῆς
δίκης
ἢ
ἀντ᾽
ἐμοῦ
γράφηται
σέ,
αὐτὰ
ταῦτα
λέγειν
ἐν
τῷ
δικαστηρίῳ
ἃ
προυκαλούμην\footnote{\parse{imp act ind 1st sg} of προκαλέω.}
αὐτόν;

\speaker{Εὐθύφρων}
ναὶ
μὰ
Δία,
ὦ
Σώκρατες,
εἰ
ἄρα
ἐμὲ
ἐπιχειρήσειε\footnote{ἐπιχειρέω - I attempt}
\versification{[5c]}
γράφεσθαι,
εὕροιμ᾽\footnote{\parse{aor opt act 1st sg} of εὑρίσκω}
ἄν,
ὡς
οἶμαι,
ὅπῃ
σαθρός\footnote{σαθρός - unsound, weak}
ἐστιν,
καὶ
πολὺ
ἂν
ἡμῖν
πρότερον
περὶ
ἐκείνου
λόγος\footnote{Here, "case".}
ἐγένετο
ἐν
τῷ
δικαστηρίῳ
ἢ
περὶ
ἐμοῦ.

\speaker{Σωκράτης}
καὶ
ἐγώ
τοι,\footnote{Αnother instance of the enclitic τοί, as we saw in 5a, meaning "surely, doubtless". Smyth section 2984 is helpful.}
ὦ
φίλε
ἑταῖρε,
ταῦτα
γιγνώσκων
μαθητὴς
ἐπιθυμῶ
γενέσθαι
σός,
εἰδὼς
ὅτι
καὶ
ἄλλος
πού
τις
καὶ
ὁ
Μέλητος
οὗτος
σὲ
μὲν
οὐδὲ
δοκεῖ\footnote{Bailly and Burnet (similarly Emlyn-Jones) takes this as "pretend" (cf. LSJ δοκέω I.4), implying avoidance because Euthyphro would be a fierce opponent. Jowett and Fowler take it as "appears/seems" (cf. LSJ δοκέω II), implying that no one seems to take much account of Euthyphro.}
ὁρᾶν,
ἐμὲ
δὲ
οὕτως
ὀξέως
[ἀτεχνῶς]
καὶ
ῥᾳδίως
κατεῖδεν\footnote{κατεῖδον - look down, observe}
ὥστε
ἀσεβείας
ἐγράψατο.
νῦν
οὖν
πρὸς
Διὸς
λέγε
μοι
ὃ
νυνδὴ
σαφῶς
εἰδέναι
διισχυρίζου,\footnote{\parse{imperf mid ind 2nd sg} of διισχυρίζομαι - I affirm confidently}
ποῖόν
τι
τὸ
εὐσεβὲς
φῂς
εἶναι
καὶ
τὸ
ἀσεβὲς
\versification{[5d]}
καὶ\footnote{TODO: both...and}
περὶ
φόνου
καὶ
περὶ
τῶν
ἄλλων;\footnote{Socrates is asking for a general definition, not just something that would apply to murder.}
ἢ
οὐ
ταὐτόν
ἐστιν
ἐν
πάσῃ
πράξει
τὸ
ὅσιον
αὐτὸ
αὑτῷ,
καὶ
τὸ
ἀνόσιον
αὖ
τοῦ
μὲν
ὁσίου
παντὸς
ἐναντίον,\footnote{τοῦ μὲν ὁσίου παντὸς ἐναντίον, "opposite of every holy thing"}
αὐτὸ
δὲ
αὑτῷ
ὅμοιον
καὶ
ἔχον
μίαν
τινὰ
ἰδέαν\footnote{ἔχον μίαν τινὰ ἰδέαν - "having a single/core idea"}
κατὰ
τὴν
ἀνοσιότητα
πᾶν
ὅτιπερ
ἂν
μέλλῃ
ἀνόσιον
εἶναι;

\speaker{Εὐθύφρων}
πάντως
δήπου,\footnote{δήπου - of course, we know}
ὦ
Σώκρατες.

\speaker{Σωκράτης}
λέγε
δή,
τί
φῂς
εἶναι
τὸ
ὅσιον
καὶ
τί
τὸ
ἀνόσιον;

\speaker{Εὐθύφρων}
λέγω
τοίνυν
ὅτι
τὸ
μὲν
ὅσιόν
ἐστιν
ὅπερ\footnote{ὅπερ - the very one which}
ἐγὼ
νῦν
ποιῶ,
τῷ
ἀδικοῦντι
ἢ\footnote{TODO: correlative}
περὶ
φόνους
ἢ
περὶ
ἱερῶν\footnote{here rendered "consecrated property" by Burnet}
κλοπὰς\footnote{κλοπή - theft}
ἤ
τι
ἄλλο
τῶν
τοιούτων
ἐξαμαρτάνοντι
ἐπεξιέναι,\footnote{ἐπίξειμι - I go out against, or in a legal context, prosecute. Everything from τῷ ἀδικοῦντι to here is governed by this infinitive.}
ἐάντε
πατὴρ
\versification{[5e]}
ὢν
τυγχάνῃ\footnote{τυγχάνω + participle of εἰμί has occurred several times so far in this book. Often the two together expresses either coincidence or is essentially quivalent to just εἰμί (LSJ A.II). Here it is the former.}
ἐάντε
μήτηρ
ἐάντε
ἄλλος
ὁστισοῦν\footnote{ὁστισοῦν = nom. sg. of ὅστις},
τὸ
δὲ
μὴ
ἐπεξιέναι
ἀνόσιον·\footnote{This last clause summarizes the previous point: "not prosecuting (him is) unholy."}
ἐπεί,
ὦ
Σώκρατες,
θέασαι\footnote{Impv.}
ὡς
μέγα
σοι
ἐρῶ
τεκμήριον\footnote{τεκμήριον - a sign, certain proof}
τοῦ
νόμου
ὅτι
οὕτως
ἔχει
--
ὃ\footnote{Referring to τεκμήριον.}
καὶ
ἄλλοις
ἤδη
εἶπον,
ὅτι
ταῦτα
ὀρθῶς
ἂν
εἴη\footnote{\parse{pres opt act 3rd sg} of εἰμί}
οὕτω
γιγνόμενα
--
μὴ
ἐπιτρέπειν\footnote{ἐπιτρέπω - here "give up, yield, give way" (cf. LSJ ἐπιτρέπω II).}
τῷ
ἀσεβοῦντι\footnote{The idea of "to go unprosecuted" is implied here, from the μὴ ἐπεξιέναι just before.}
μηδ᾽
ἂν
ὁστισοῦν
τυγχάνῃ
ὤν.
αὐτοὶ\footnote{What follows in the τεκμήριον mentioned above.}
γὰρ
οἱ
ἄνθρωποι
τυγχάνουσι\footnote{Expressing coincidence again, "coincidentally they already think that...".}
νομίζοντες
τὸν
Δία
τῶν
θεῶν
ἄριστον
καὶ
δικαιότατον,
\versification{[6a]}
καὶ
τοῦτον
ὁμολογοῦσι
τὸν
αὑτοῦ
πατέρα\footnote{Χρόνος}
δῆσαι\footnote{Here referring to Zeus' activities against his father. See Hesiod, Theogony, Theogony, 453-492.}
ὅτι
τοὺς
ὑεῖς\footnote{ὑεῖς = υἱούς}
κατέπινεν\footnote{καταπίνω - swallow}
οὐκ
ἐν
δίκῃ,\footnote{ἐν δίκῃ is adverbially, so "not rightly". Similarly, see usage in 4b.}
κἀκεῖνόν\footnote{Referring back to πατέρα, not Zeus, because it was Chronos who castrated his father. So Euthyphro is working backwards in time. Hesiod, Theogony, 176-206}
γε
αὖ
τὸν
αὑτοῦ
πατέρα\footnote{Οὐρανός}
ἐκτεμεῖν
δι᾽
ἕτερα
τοιαῦτα·
ἐμοὶ
δὲ
χαλεπαίνουσιν\footnote{χαλεπαίνω - I am severe, sore, grievous}
ὅτι
τῷ
πατρὶ
ἐπεξέρχομαι
ἀδικοῦντι,
καὶ
οὕτως
αὐτοὶ
αὑτοῖς
τὰ
ἐναντία\footnote{ἐναντία - opposite, contrary}
λέγουσι
περί
τε
τῶν
θεῶν
καὶ
περὶ
ἐμοῦ.




%========================================== SECTION 6 ==========================================%




\speaker{Σωκράτης}
ἆρά
γε,
ὦ
Εὐθύφρων,
τοῦτ᾽
ἔστιν
[οὗ]
οὕνεκα\footnote{οὕνεκα - on which account, therefore}
τὴν
γραφὴν
φεύγω,\footnote{φευγω - in most cases "I flee" or something similar. However, in legal contexts (like this), it can mean "I am accused/prosecuted".}
ὅτι
τὰ
τοιαῦτα
ἐπειδάν\footnote{ἐπειδάν - whenever}
τις
περὶ
τῶν
θεῶν
λέγῃ,
δυσχερῶς\footnote{δυσχερής, -ές - difficult, annoying. Emlyn-Jones, Burnet, go with the former meaning. LSJ goes with the latter meaning here.}
πως
ἀποδέχομαι;\footnote{"I understand" according to LSJ. Taking δυσχερῶς πως ἀποδέχομαι together, perhaps "I accept with difficulty" would be better?}
διὸ
δή,
ὡς
ἔοικε,
φήσει\footnote{\parse{aor act subj 3rd sg} of φήμι}
τίς
με
ἐξαμαρτάνειν.
νῦν
οὖν
εἰ
καὶ
σοὶ
ταῦτα
συνδοκεῖ\footnote{συνδοκέω - to seem good also}
τῷ 
\versification{[6b]}
εὖ
εἰδότι\footnote{\parse{perf act dat sg ptcpl} of οἶδα}
περὶ
τῶν
τοιούτων,
ἀνάγκη
δή,
ὡς
ἔοικε,
καὶ
ἡμῖν
συγχωρεῖν.\footnote{συγχωρέω - I come together, meet; concede or grant in an argument}
τί
γὰρ
καὶ
φήσομεν,
οἵ
γε
καὶ
αὐτοὶ
ὁμολογοῦμεν\footnote{Burnet and Fowler both translate this and φήσομεν as 1st person singular. On what basis?}
περὶ
αὐτῶν
μηδὲν
εἰδέναι;
ἀλλά
μοι
εἰπὲ
πρὸς
Φιλίου,
σὺ
ὡς
ἀληθῶς
ἡγῇ\footnote{\parse{pres mid subj 2nd sg} of ἡγέομαι - believe, hold}
ταῦτα
οὕτως
γεγονέναι;

\speaker{Εὐθύφρων}
καὶ
ἔτι
γε
τούτων
θαυμασιώτερα,\footnote{Comp. of θαυμάσιος - wonderful, marvelous}
ὦ
Σώκρατες,
ἃ
οἱ
πολλοὶ
οὐκ
ἴσασιν.\footnote{\parse{perf act ind 3rd pl} of οἶδα}

\speaker{Σωκράτης}
καὶ
πόλεμον\footnote{πόλεμος - war}
ἆρα
ἡγῇ\footnote{\parse{pres mid subj 2nd sg} of ἡγέομαι - believe, hold}
σὺ
εἶναι
τῷ
ὄντι
ἐν
τοῖς
θεοῖς
πρὸς
ἀλλήλους,
καὶ
ἔχθρας\footnote{ἔχθρα - hatred, enmity}
γε
δεινὰς\footnote{δεινός - fearful, terrible}
καὶ
μάχας\footnote{μάχη - battle}
καὶ
ἄλλα
τοιαῦτα
πολλά,
οἷα
λέγεταί
τε
ὑπὸ
τῶν
ποιητῶν,
καὶ
ὑπὸ
τῶν 
\versification{[6c]}
ἀγαθῶν
γραφέων\footnote{γραφεύς - writer, painter}
τά
τε
ἄλλα
ἱερὰ
ἡμῖν
καταπεποίκιλται,\footnote{ποικίλλω - deck with various colors or in diverse modes, mottle}
καὶ
δὴ
καὶ
τοῖς
μεγάλοις
Παναθηναίοις
ὁ
πέπλος\footnote{πέπλος - woven cloth. In Athens in the Panathenaia, it was the robe carried in the procession for the statue of Athena.}
μεστὸς
τῶν
τοιούτων
ποικιλμάτων\footnote{ποίκιλμα - embroidery; variety, diversity}
ἀνάγεται
εἰς
τὴν
ἀκρόπολιν;
ταῦτα
ἀληθῆ
φῶμεν
εἶναι,
ὦ
Εὐθύφρων;

\speaker{Εὐθύφρων}
μὴ 
μόνον
γε,
ὦ
Σώκρατες,
ἀλλ᾽
ὅπερ
ἄρτι
εἶπον,
καὶ
ἄλλα
σοι
ἐγὼ
πολλά,
ἐάνπερ
βούλῃ,\footnote{\parse{pr act subj 2nd sg} of βούλομαι}
περὶ
τῶν
θείων
διηγήσομαι,
ἃ
σὺ
ἀκούων
εὖ
οἶδ᾽
ὅτι
ἐκπλαγήσῃ.\footnote{Either from ἐκπλάσσω meaning "to model exactly" or from ἐκπλήσσω meaning "to amaze, astound". Both could make sense here but given Socrates response, I would lean with the latter.}

\speaker{Σωκράτης}
οὐκ
ἂν
θαυμάζοιμι.
ἀλλὰ
ταῦτα
μέν
μοι
εἰς
αὖθις\footnote{αὖθις - back, back again. εἰς αὖθις together means "another time", Burnet, 116.}
ἐπὶ
σχολῆς\footnote{σχολή - leisure}
διηγήσῃ·
νυνὶ
δὲ
ὅπερ
ἄρτι
σε
ἠρόμην\footnote{\parse{aor mid ind 1st sg} of ἔρομαι - I ask}
πειρῶ\footnote{\parse{pr act impv 2nd sg} of πειράω - attempt, endeavor, try.}
\versification{[6d]}
σαφέστερον\footnote{comp. of σαφής}
εἰπεῖν.
οὐ
γάρ
με,
ὦ
ἑταῖρε,
τὸ
πρότερον
ἱκανῶς
ἐδίδαξας
ἐρωτήσαντα
τὸ
ὅσιον
ὅτι
ποτ᾽
εἴη,
ἀλλά
μοι
εἶπες
ὅτι
τοῦτο
τυγχάνει
ὅσιον
ὂν
ὃ
σὺ
νῦν
ποιεῖς,
φόνου
ἐπεξιὼν\footnote{\parse{pr act ptcpl} of ἐπέξειμι}
τῷ
πατρί.

\speaker{Εὐθύφρων}
καὶ
ἀληθῆ
γε
ἔλεγον,
ὦ
Σώκρατες.

\speaker{Σωκράτης}
ἴσως.
ἀλλὰ
γάρ,
ὦ
Εὐθύφρων,
καὶ
ἄλλα
πολλὰ
φῂς
εἶναι
ὅσια.

\speaker{Εὐθύφρων}
καὶ
γὰρ
ἔστιν.

\speaker{Σωκράτης}
μέμνησαι\footnote{\parse{perf mp ind 2nd sg} of μιμνήσκω - remind}
οὖν
ὅτι
οὐ
τοῦτό
σοι
διεκελευόμην,\footnote{διακελεύομαι - I give orders, direct, exhort}
ἕν
τι
ἢ
δύο
με
διδάξαι
τῶν
πολλῶν
ὁσίων,
ἀλλ᾽
ἐκεῖνο
αὐτὸ
τὸ
εἶδος\footnote{εἶδος - form, kind, nature. Either nom. or acc. I think nominative in a predicate position with ἐκεῖνο.}
ᾧ
πάντα
τὰ
ὅσια
ὅσιά
ἐστιν;
ἔφησθα\footnote{\parse{imp ind act 2nd sg} of φήμι}
γάρ
που
μιᾷ
ἰδέᾳ
\versification{[6e]}
τά
τε
ἀνόσια
ἀνόσια
εἶναι
καὶ
τὰ
ὅσια
ὅσια·
ἢ
οὐ
μνημονεύεις;

\speaker{Εὐθύφρων}
ἔγωγε.

\speaker{Σωκράτης}
ταύτην
τοίνυν\footnote{τοίνυν - therefore, accordingly}
με
αὐτὴν
δίδαξον
τὴν
ἰδέαν
τίς
ποτέ
ἐστιν,
ἵνα
εἰς
ἐκείνην
ἀποβλέπων\footnote{ἀποβλέπω - look away from; pay attention to}
καὶ
χρώμενος\footnote{\parse{pres mp ptcpl} of χράω. Perhaps "using this model" is the meaning?}
αὐτῇ
παραδείγματι,\footnote{παράδειγμα - model, pattern}
ὃ
μὲν
ἂν
τοιοῦτον
ᾖ\footnote{\parse{pres act subj 3rd sg} of εἰμί}
ὧν
ἂν
ἢ
σὺ
ἢ
ἄλλος
τις
πράττῃ
φῶ\footnote{\parse{pres act subj 1st sg} of φήμι}
ὅσιον
εἶναι,
ὃ
δ᾽
ἂν
μὴ
τοιοῦτον,
μὴ
φῶ.

\speaker{Εὐθύφρων}
ἀλλ᾽
εἰ
οὕτω
βούλει,
ὦ
Σώκρατες,
καὶ
οὕτω
σοι
φράσω.\footnote{φράζω - I point out, show}

\speaker{Σωκράτης}
ἀλλὰ
μὴν
βούλομαί
γε.

\speaker{Εὐθύφρων}
ἔστι
τοίνυν
τὸ
μὲν
τοῖς
θεοῖς
προσφιλὲς\footnote{προσφιλής - dear, loved}
ὅσιον,
τὸ
\versification{[7a]}
δὲ
μὴ
προσφιλὲς
ἀνόσιον.



%========================================== SECTION 7 ==========================================%



\speaker{Σωκράτης}
παγκάλως,\footnote{παγκάλως - all, very good, right}
ὦ
Εὐθύφρων,
καὶ
ὡς
ἐγὼ
ἐζήτουν
ἀποκρίνασθαί
σε,
οὕτω
νῦν
ἀπεκρίνω.\footnote{Note, this is \parse{aor mid ind 2nd sg}, not \parse{pres act ind 1st sg}!}
εἰ
μέντοι
ἀληθῶς,
τοῦτο
οὔπω
οἶδα,
ἀλλὰ
σὺ
δῆλον
ὅτι
ἐπεκδιδάξεις
ὡς
ἔστιν
ἀληθῆ
ἃ
λέγεις.

\speaker{Εὐθύφρων}
πάνυ
μὲν
οὖν.\footnote{οὖν μὲν πάνυ - by all means (see LSJ πάνυ A.4)}

\speaker{Σωκράτης}
φέρε\footnote{Perhaps with the meaning in LSJ φέρω A.IX.2, "come"}
δή,
ἐπισκεψώμεθα
τί
λέγομεν.
τὸ
μὲν
θεοφιλές\footnote{θεοφιλής, -ές - dear to the gods}
τε
καὶ
θεοφιλὴς
ἄνθρωπος
ὅσιος,
τὸ
δὲ
θεομισὲς\footnote{θεομισής, -, ές - hated by the gods}
καὶ
ὁ
θεομισὴς
ἀνόσιος·
οὐ
ταὐτὸν\footnote{ταὐτός, -ή, -όν - identical.}
δ᾽
ἐστίν,
ἀλλὰ
τὸ
ἐναντιώτατον,
τὸ
ὅσιον
τῷ
ἀνοσίῳ·
οὐχ
οὕτως;

\speaker{Εὐθύφρων}
οὕτω
μὲν
οὖν.

\speaker{Σωκράτης}
καὶ
εὖ
γε
φαίνεται
εἰρῆσθαι; 

\speaker{Εὐθύφρων}
\versification{[7b]}
δοκῶ,
ὦ
Σώκρατες.
[εἴρηται
γάρ.]

\speaker{Σωκράτης}
οὐκοῦν
καὶ
ὅτι
στασιάζουσιν\footnote{στασιάζω - I am at variance, disagree, quarrel, be in a state of disagreement}
οἱ
θεοί,
ὦ
Εὐθύφρων,
καὶ
διαφέρονται
ἀλλήλοις
καὶ
ἔχθρα
ἐστὶν
ἐν
αὐτοῖς
πρὸς
ἀλλήλους,
καὶ
τοῦτο
εἴρηται;

\speaker{Εὐθύφρων}
εἴρηται
γάρ.

\speaker{Σωκράτης}
ἔχθραν
δὲ
καὶ
ὀργάς,
ὦ
ἄριστε,
ἡ
περὶ
τίνων
διαφορὰ
ποιεῖ;\footnote{Jowett translates "what sort of difference creates...". Fowler translates "what things is the disagreement about, which causes..." In this case the key bit is the περὶ τίνων because the question is about what causes fights among the gods.}
ὧδε\footnote{Here expressing manner, such as "let's look at it this way...". It's the setup for the example that follows.}
δὲ
σκοπῶμεν.
ἆρ᾽
ἂν
εἰ
διαφεροίμεθα
ἐγώ
τε
καὶ
σὺ
περὶ
ἀριθμοῦ
ὁπότερα\footnote{ὁπότερος - which of two}
πλείω,\footnote{neut. pl. of πλείων. Burnet clarifies this question is about which of two groups is greater, not which of two numbers is greater. However Jowett and Fowler translate it as if they would disagree. It seems to me that Burnet is probably right. Which of two numbers is greater should be obvious. But if you have to groups that need to be counted to determine which is larger, you have a more legitimate question.}
ἡ
περὶ
τούτων
διαφορὰ
ἐχθροὺς
ἂν
ἡμᾶς
ποιοῖ
καὶ
ὀργίζεσθαι
ἀλλήλοις,
ἢ
ἐπὶ
λογισμὸν\footnote{λογισμός - calculation}
ἐλθόντες
περί
γε
τῶν
τοιούτων
ταχὺ
ἂν
\versification{[7c]}
ἀπαλλαγεῖμεν;\footnote{ἀπαλλάσσω - I put away, release, discharge}

\speaker{Εὐθύφρων}
πάνυ
γε.

\speaker{Σωκράτης}
οὐκοῦν
καὶ
περὶ
τοῦ
μείζονος
καὶ
ἐλάττονος
εἰ
διαφεροίμεθα,
ἐπὶ
τὸ
μετρεῖν
ἐλθόντες
ταχὺ
παυσαίμεθ᾽\footnote{\parse{aor mid opt 1st pl} of παύω}
ἂν
τῆς
διαφορᾶς;

\speaker{Εὐθύφρων}
ἔστι
ταῦτα.

\speaker{Σωκράτης}
καὶ
ἐπί
γε
τὸ
ἱστάναι\footnote{here, "weighing".}
ἐλθόντες,
ὡς
ἐγᾦμαι,\footnote{ἐγώ + οἴομαι}
περὶ
τοῦ
βαρυτέρου\footnote{βαρύς - heavy, weighty, difficult}
τε
καὶ
κουφοτέρου\footnote{κοῦφος, -η, -ον - easy, light, nimble}
διακριθεῖμεν\footnote{διακρίνω - I decide}
ἄν;

\speaker{Εὐθύφρων}
πῶς
γὰρ
οὔ;

\speaker{Σωκράτης}
περὶ
τίνος
δὲ
δὴ
διενεχθέντες
καὶ
ἐπὶ
τίνα
κρίσιν\footnote{Burnet makes a big deal out of this word meaning "settlement" versus agreement but I don't understand the distinction he's making.}
οὐ
δυνάμενοι
ἀφικέσθαι
ἐχθροί
γε
ἂν
ἀλλήλοις
εἶμεν
καὶ
ὀργιζοίμεθα;
ἴσως\footnote{"perhaps"?}
οὐ
πρόχειρόν\footnote{πρόχειρος - at hand. The idea is that "perhaps you can't think of an example on the spot".}
σοί
ἐστιν,
ἀλλ᾽
ἐμοῦ
λέγοντος\footnote{Genitive absolute? As for the meaning, Jowett translates ἵσως...λέγοντος more negatively, "I dare say that the answer does not occur to you at the moment" while Fowler has Socrates cut Euthyphro some slack, "Perhaps you cannot give an answer offhand; but let me suggest it."}
\versification{[7d]}
σκόπει\footnote{\parse{pres act impv 2nd sg}}
εἰ
τάδε
ἐστὶ
τό
τε
δίκαιον
καὶ
τὸ
ἄδικον
καὶ
καλὸν
καὶ
αἰσχρὸν
καὶ
ἀγαθὸν
καὶ
κακόν.
ἆρα
οὐ
ταῦτά
ἐστιν\footnote{I'm not sure grammatically what is going on here.}
περὶ
ὧν
διενεχθέντες
καὶ
οὐ
δυνάμενοι
ἐπὶ
ἱκανὴν
κρίσιν
αὐτῶν
ἐλθεῖν
ἐχθροὶ
ἀλλήλοις
γιγνόμεθα,
ὅταν
γιγνώμεθα,
καὶ
ἐγὼ
καὶ
σὺ
καὶ
οἱ
ἄλλοι
ἄνθρωποι
πάντες;

\speaker{Εὐθύφρων}
ἀλλ᾽
ἔστιν
αὕτη
ἡ
διαφορά,
ὦ
Σώκρατες,
καὶ
περὶ
τούτων.

\speaker{Σωκράτης}
τί
δὲ
οἱ
θεοί,
ὦ
Εὐθύφρων;
οὐκ
εἴπερ
τι
διαφέρονται,
δι᾽
αὐτὰ
ταῦτα
διαφέροιντ᾽
ἄν;

\speaker{Εὐθύφρων}
πολλὴ
ἀνάγκη. 

\speaker{Σωκράτης}
\versification{[7e]}
καὶ
τῶν
θεῶν
ἄρα,
ὦ
γενναῖε\footnote{γενναῖος - excellent}
Εὐθύφρων,
ἄλλοι
ἄλλα\footnote{Burnet seems to imply that this is correlating with ἄλλοι in some sense of "some gods...other gods" but that seems strange to me. Seems more naturally to refer to the things they think.}
δίκαια
ἡγοῦνται
κατὰ
τὸν
σὸν
λόγον,
καὶ
καλὰ
καὶ
αἰσχρὰ
καὶ
ἀγαθὰ
καὶ
κακά·
οὐ
γὰρ
ἄν
που\footnote{doubtless}
ἐστασίαζον
ἀλλήλοις
εἰ
μὴ
περὶ
τούτων
διεφέροντο·
ἦ
γάρ;

\speaker{Εὐθύφρων}
ὀρθῶς
λέγεις.

\speaker{Σωκράτης}
οὐκοῦν\footnote{οὐκοῦν - therefore}
ἅπερ\footnote{ὅσπερ = ὅς}
καλὰ
ἡγοῦνται
ἕκαστοι
καὶ
ἀγαθὰ
καὶ
δίκαια,
ταῦτα
καὶ
φιλοῦσιν,
τὰ
δὲ
ἐναντία
τούτων
μισοῦσιν;

\speaker{Εὐθύφρων}
πάνυ
γε.

\speaker{Σωκράτης}
ταὐτὰ
δέ
γε,
ὡς
σὺ
φῄς,
οἱ\footnote{"οἱ...οἱ", some...others}
μὲν
δίκαια
ἡγοῦνται,
\versification{[8a]}
οἱ
δὲ
ἄδικα,
περὶ
ἃ
καὶ
ἀμφισβητοῦντες\footnote{ἀμφισβητέω - I disagree, stand apart}
στασιάζουσί
τε
καὶ
πολεμοῦσιν
ἀλλήλοις·
ἆρα
οὐχ
οὕτω;



%========================================== SECTION 8 ==========================================%



\speaker{Εὐθύφρων}
οὕτω.

\speaker{Σωκράτης}
ταὔτ᾽
ἄρα,
ὡς
ἔοικεν,
μισεῖταί
τε
ὑπὸ
τῶν
θεῶν
καὶ
φιλεῖται,
καὶ
θεομισῆ
τε
καὶ
θεοφιλῆ
ταὔτ᾽
ἂν
εἴη.

\speaker{Εὐθύφρων}
ἔοικεν.

\speaker{Σωκράτης}
καὶ
ὅσια
ἄρα
καὶ
ἀνόσια
τὰ
αὐτὰ
ἂν
εἴη,
ὦ
Εὐθύφρων,
τούτῳ
τῷ
λόγῳ.

\speaker{Εὐθύφρων}
κινδυνεύει.

\speaker{Σωκράτης}
οὐκ
ἄρα
ὃ
ἠρόμην\footnote{\parse{aor mid ind 1st sg} of ἔρομαι - I ask}
ἀπεκρίνω,
ὦ
θαυμάσιε.
οὐ
γὰρ
τοῦτό
γε
ἠρώτων,
ὃ
τυγχάνει
ταὐτὸν
ὂν
ὅσιόν
τε
καὶ
ἀνόσιον·
ὃ
δ᾽
ἂν
θεοφιλὲς
ᾖ
καὶ
θεομισές
ἐστιν,
ὡς
ἔοικεν.
\versification{[8b]}
ὥστε,
ὦ
Εὐθύφρων,
ὃ
σὺ
νῦν
ποιεῖς
τὸν
πατέρα
κολάζων,\footnote{κολάζω - chastise, punish}
οὐδὲν
θαυμαστὸν
εἰ
τοῦτο
δρῶν
τῷ
μὲν
Διὶ
προσφιλὲς
ποιεῖς,
τῷ
δὲ
Κρόνῳ
καὶ
τῷ
Οὐρανῷ
ἐχθρόν,
καὶ
τῷ
μὲν
Ἡφαίστῳ
φίλον,
τῇ
δὲ
Ἥρᾳ
ἐχθρόν,
καὶ
εἴ
τις
ἄλλος
τῶν
θεῶν
ἕτερος
ἑτέρῳ
διαφέρεται
περὶ
αὐτοῦ,
καὶ
ἐκείνοις
κατὰ
τὰ
αὐτά.

\speaker{Εὐθύφρων}
ἀλλ᾽
οἶμαι,
ὦ
Σώκρατες,
περί
γε
τούτου
τῶν
θεῶν
οὐδένα
ἕτερον
ἑτέρῳ
διαφέρεσθαι,
ὡς
οὐ
δεῖ
δίκην
διδόναι
ἐκεῖνον
ὃς
ἂν
ἀδίκως
τινὰ
ἀποκτείνῃ.

\speaker{Σωκράτης}
τί
δέ;
ἀνθρώπων,
ὦ
Εὐθύφρων,
ἤδη
τινὸς
ἤκουσας
\versification{[8c]}
ἀμφισβητοῦντος
ὡς
τὸν
ἀδίκως
ἀποκτείναντα
ἢ
ἄλλο
ἀδίκως
ποιοῦντα
ὁτιοῦν\footnote{ὁτιοῦν - whatever}
οὐ
δεῖ
δίκην
διδόναι;











