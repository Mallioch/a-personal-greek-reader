\pagebreak

\sect{Plato's Euthyphro}

\speaker{Εὐθύφρων}
\versification{[2a]}
τί
νεώτερον,
ὦ
Σώκρατες,
γέγονεν,
ὅτι
σὺ
τὰς
ἐν
Λυκείῳ
καταλιπὼν
διατριβὰς\footnote{διατριβή - a way of spending time}
ἐνθάδε\footnote{ἐνθάδε - here}
νῦν
διατρίβεις
περὶ
τὴν
τοῦ
βασιλέως
στοάν;
οὐ
γάρ
που\footnote{"For certainly there is no...is there?" See discussion in Smyth 2651 and FCC on this.}
καὶ
σοί
γε
δίκη
τις
οὖσα
τυγχάνει
πρὸς
τὸν
βασιλέα
ὥσπερ
ἐμοί.\footnote{Dative of the possessor. See Smyth 1476ff.}

\speaker{Σωκράτης}
οὔτοι
δὴ
Ἀθηναῖοί
γε,
ὦ
Εὐθύφρων,
δίκην
αὐτὴν
καλοῦσιν
ἀλλὰ
γραφήν.\footnote{Moving from generic to specific, δίκη (more general word for case) to a γραφή (a case about actions against the state)}

\speaker{Εὐθύφρων}
\versification{[2b]}
τί
φῄς;
γραφὴν
σέ
τις,
ὡς
ἔοικε,
γέγραπται·
οὐ
γὰρ
ἐκεῖνό
γε
καταγνώσομαι,
ὡς
σὺ
ἕτερον.

\speaker{Σωκράτης}
οὐ
γὰρ
οὖν.

\speaker{Εὐθύφρων}
ἀλλὰ
σὲ
ἄλλος;

\speaker{Σωκράτης}
πάνυ
γε.

\speaker{Εὐθύφρων}
τίς
οὗτος;

\speaker{Σωκράτης}
οὐδ᾽
αὐτὸς
πάνυ
τι
γιγνώσκω,
ὦ
Εὐθύφρων,
τὸν
ἄνδρα,
νέος
γάρ
τίς
μοι
φαίνεται
καὶ
ἀγνώς·
ὀνομάζουσι
μέντοι
αὐτόν,
ὡς
ἐγᾦμαι,\footnote{ἐγώ + οἶμαι. οἶμαι - I suppose, expect, imagine.}
Μέλητον.
ἔστι
δὲ
τῶν
δήμων\footnote{δῆμος - people}
Πιτθεύς,
εἴ
τινα
νῷ\footnote{νόος - mind}
ἔχεις
Πιτθέα
Μέλητον
οἷον
τετανότριχα\footnote{τετανότριξ - having long hair}
καὶ
οὐ
πάνυ
εὐγένειον,\footnote{εὐγένειος - well-maned (referring here to his beard?)}
ἐπίγρυπον\footnote{ἐπίγρυπος - somewhat-hooked (nose)}
δέ.

\speaker{Εὐθύφρων}
οὐκ
ἐννοῶ,\footnote{ἐννοέω - I think, consider}
ὦ
Σώκρατες·
ἀλλὰ
δὴ
τίνα
γραφήν
\versification{[2c]}
σε
γέγραπται;

\speaker{Σωκράτης}
ἥντινα;
οὐκ
ἀγεννῆ,\footnote{ἀγεννή - minor}
ἔμοιγε\footnote{dat. sg. of ἐγώ}
δοκεῖ·
τὸ
γὰρ
νέον
ὄντα
τοσοῦτον
πρᾶγμα\footnote{probably "lawsuit" here.}
ἐγνωκέναι
οὐ
φαῦλόν\footnote{φαῦλος - small, paltry}
ἐστιν.
ἐκεῖνος
γάρ,
ὥς
φησιν,
οἶδε
τίνα
τρόπον
οἱ
νέοι
διαφθείρονται
καὶ
τίνες
οἱ
διαφθείροντες
αὐτούς.
καὶ
κινδυνεύει\footnote{"It might probably be" here, a seemingly less-common usage.}
σοφός
τις
εἶναι,
καὶ
τὴν
ἐμὴν
ἀμαθίαν\footnote{ἀμαθία - ignorance}
κατιδὼν\footnote{κατεῖδον - I look down; observe}
ὡς
διαφθείροντος
τοὺς
ἡλικιώτας\footnote{ἡλικιίωτης - an equal in age, fellow, comrade}
αὐτοῦ,
ἔρχεται
κατηγορήσων
μου
ὥσπερ
πρὸς
μητέρα\footnote{As though the city were his mother}
πρὸς
τὴν
πόλιν.
καὶ
φαίνεταί
μοι
τῶν
πολιτικῶν\footnote{πολιτικός - of or relating to citizens; public; official; statesman; belonging to the state; common, ordinary}
\versification{[2d]}
μόνος
ἄρχεσθαι
ὀρθῶς·
ὀρθῶς
γάρ
ἐστι
τῶν
νέων
πρῶτον
ἐπιμεληθῆναι\footnote{ἐπιμελέομαι - I take care of, have charge of; have the management of}
ὅπως
ἔσονται
ὅτι
ἄριστοι,
ὥσπερ
γεωργὸν
ἀγαθὸν
τῶν
νέων
φυτῶν\footnote{φυτόν - a plant, tree}
εἰκὸς
πρῶτον
ἐπιμεληθῆναι,
μετὰ
δὲ
τοῦτο
καὶ
τῶν
ἄλλων.
καὶ
δὴ
καὶ
Μέλητος
ἴσως\footnote{ἴσως - equally; perhaps; it may be that}
πρῶτον
\versification{[3a]}
μὲν
ἡμᾶς
ἐκκαθαίρει\footnote{ἐκκαθαίρω - I clean out, clean thoroughly}
τοὺς
τῶν
νέων
τὰς
βλάστας\footnote{βλάστη - shoot}
διαφθείροντας,
ὥς
φησιν·
ἔπειτα
μετὰ
τοῦτο
δῆλον
ὅτι
τῶν
πρεσβυτέρων
ἐπιμεληθεὶς
πλείστων
καὶ
μεγίστων
ἀγαθῶν
αἴτιος
τῇ
πόλει
γενήσεται,
ὥς
γε
τὸ
εἰκὸς
συμβῆναι\footnote{συμβαίνω - come together; come to pass}
ἐκ
τοιαύτης
ἀρχῆς
ἀρξαμένῳ. 






\speaker{Εὐθύφρων}
βουλοίμην
ἄν,
ὦ
Σώκρατες,
ἀλλ᾽
ὀρρωδῶ\footnote{ὀρρωδέω - I fear, dread, shrink from}
μὴ
τοὐναντίον\footnote{"the opposite"}
γένηται·
ἀτεχνῶς\footnote{ἀτεχνῶς - simply}
γάρ
μοι
δοκεῖ
ἀφ᾽
ἑστίας\footnote{ἑστία - the hearth, fireside. Idiomatically, "at its very heart". Meletos is beginning at the heart of the city to do wrong.}
ἄρχεσθαι
κακουργεῖν\footnote{κακουργέω - I do evil, work wickedness}
τὴν
πόλιν,
ἐπιχειρῶν
ἀδικεῖν
σέ.
καί
μοι
λέγε,
τί
καὶ
ποιοῦντά
σέ
φησι
διαφθείρειν
τοὺς
νέους; 

\speaker{Σωκράτης}
\versification{[3b]}
ἄτοπα,\footnote{ἄτοπος - improper, unrighteous, perverse}
ὦ
θαυμάσιε,\footnote{θαυμάσιος - wonderful, admirable}
ὡς
οὕτω
γ᾽
ἀκοῦσαι.\footnote{"to hear it just like that"}
φησὶ
γάρ
με
ποιητὴν
εἶναι
θεῶν,
καὶ
ὡς
καινοὺς
ποιοῦντα
θεοὺς
τοὺς
δ᾽
ἀρχαίους
οὐ
νομίζοντα
ἐγράψατο
τούτων
αὐτῶν
ἕνεκα,
ὥς
φησιν.

\speaker{Εὐθύφρων}
μανθάνω,
ὦ
Σώκρατες·
ὅτι
δὴ
σὺ
τὸ
δαιμόνιον
φῂς
σαυτῷ
ἑκάστοτε\footnote{ἑκάστοτε - each time, always}
γίγνεσθαι.
ὡς
οὖν
καινοτομοῦντός\footnote{καινοτομέω - I cut fresh into; innovate}
σου
περὶ
τὰ
θεῖα
γέγραπται
ταύτην
τὴν
γραφήν,
καὶ
ὡς
διαβαλῶν\footnote{διαβάλλω - I slander}
δὴ
ἔρχεται
εἰς
τὸ
δικαστήριον,\footnote{δικαστήριον - a court of justice}
εἰδὼς
ὅτι
εὐδιάβολα\footnote{εὐδιάβολος - easy to misrepresent}
τὰ
τοιαῦτα
πρὸς
τοὺς
πολλούς.
καὶ
ἐμοῦ
γάρ
τοι,
\versification{[3c]}
ὅταν
τι
λέγω
ἐν
τῇ
ἐκκλησίᾳ
περὶ
τῶν
θείων,
προλέγων
αὐτοῖς
τὰ
μέλλοντα,
καταγελῶσιν\footnote{\parse{pr act ind 3rd pl} καταγελάω - I deride, laugh scornfully at.}
ὡς
μαινομένου·
καίτοι
οὐδὲν
ὅτι
οὐκ
ἀληθὲς
εἴρηκα
ὧν
προεῖπον,
ἀλλ᾽
ὅμως
φθονοῦσιν\footnote{φθονέω - I envy}
ἡμῖν
πᾶσι
τοῖς
τοιούτοις.
ἀλλ᾽
οὐδὲν
αὐτῶν
χρὴ
φροντίζειν,
ἀλλ᾽
ὁμόσε\footnote{ὁμόσε - to one and the same place}
ἰέναι.\footnote{\parse{pres act inf} of εἶμι, I go. "Close with the enemy" according to Emlyn-Jones.}

\speaker{Σωκράτης}
ὦ
φίλε
Εὐθύφρων,
ἀλλὰ
τὸ
μὲν
καταγελασθῆναι
ἴσως
οὐδὲν
πρᾶγμα.\footnote{"Is of no consequence", Emlyn-Jones}
Ἀθηναίοις
γάρ
τοι,
ὡς
ἐμοὶ
δοκεῖ,
οὐ
σφόδρα
μέλει\footnote{μέλω - I care for; pass: to be an object of care}
ἄν
τινα
δεινὸν
οἴωνται\footnote{οἴομαι - I suppose, expect, imagine}
εἶναι,
μὴ
μέντοι
διδασκαλικὸν\footnote{διδασκαλικός - fit for teaching, instructive}
τῆς
αὑτοῦ
σοφίας·
ὃν
δ᾽
ἂν
καὶ
ἄλλους
οἴωνται
\versification{[3d]}
ποιεῖν
τοιούτους,
θυμοῦνται,\footnote{θυμόω - I am very angry}
εἴτ᾽
οὖν
φθόνῳ
ὡς
σὺ
λέγεις,
εἴτε
δι᾽
ἄλλο
τι.

\speaker{Εὐθύφρων}
τούτου
οὖν
πέρι
ὅπως
ποτὲ
πρὸς
ἐμὲ
ἔχουσιν,
οὐ
πάνυ
ἐπιθυμῶ\footnote{ἐπιθυμέω - I desire}
πειραθῆναι.\footnote{πειράω - I try, attempt}

\speaker{Σωκράτης}
ἴσως
γὰρ
σὺ
μὲν
δοκεῖς
σπάνιον\footnote{σπάνιος - rare, scarce, scanty}
σεαυτὸν
παρέχειν
καὶ
διδάσκειν
οὐκ
ἐθέλειν
τὴν
σεαυτοῦ
σοφίαν·
ἐγὼ
δὲ
φοβοῦμαι
μὴ
ὑπὸ
φιλανθρωπίας\footnote{φιλανθρωπία - benevolence}
δοκῶ
αὐτοῖς
ὅτιπερ
ἔχω
ἐκκεχυμένως\footnote{ἐκκεχυμένως - profusely}
παντὶ
ἀνδρὶ
λέγειν,
οὐ
μόνον
ἄνευ
μισθοῦ,
ἀλλὰ
καὶ
προστιθεὶς
ἂν
ἡδέως\footnote{ἡδέως - gladly, with pleasure}
εἴ
τίς
μου
ἐθέλει
ἀκούειν.
εἰ
μὲν
οὖν,
ὃ
νυνδὴ
ἔλεγον,
μέλλοιέν\footnote{\parse{pr act opt 3rd pl} of μέλλω}
μου
καταγελᾶν
ὥσπερ
\versification{[3e]}
σὺ
φῂς
σαυτοῦ,
οὐδὲν
ἂν
εἴη
ἀηδὲς\footnote{ἀηδής - unpleasant to the taste, distasteful}
παίζοντας\footnote{παίζω - I play}
καὶ
γελῶντας
ἐν
τῷ
δικαστηρίῳ
διαγαγεῖν·\footnote{διάγω - I pass time; carry over}
εἰ
δὲ
σπουδάσονται,
τοῦτ᾽
ἤδη
ὅπῃ\footnote{ὅπη - by which way}
ἀποβήσεται\footnote{ἀποβαίνω - (of events) to happen, turn out}
ἄδηλον\footnote{ἄδηλος - unseen, inconspicuous}
πλὴν
ὑμῖν
τοῖς
μάντεσιν.\footnote{μάντις - one who divines, seer}

\speaker{Εὐθύφρων}
ἀλλ᾽
ἴσως\footnote{ἵσως - perhaps, probably}
οὐδὲν
ἔσται,
ὦ
Σώκρατες,
πρᾶγμα,
ἀλλὰ
σύ
τε
κατὰ
νοῦν\footnote{"satisfactorily", Burnet}
ἀγωνιῇ\footnote{\parse{fut m ind 2d sg} of ἀγωνίζομαι}
τὴν
δίκην,
οἶμαι
δὲ
καὶ
ἐμὲ
τὴν
ἐμήν.

\speaker{Σωκράτης}
ἔστιν
δὲ
δὴ
σοί,
ὦ
Εὐθύφρων,
τίς
ἡ
δίκη;
φεύγεις\footnote{In the context of legal matters, φεύγω takes on the idea of defending.}
αὐτὴν
ἢ
διώκεις;

\speaker{Εὐθύφρων}
διώκω.

\speaker{Σωκράτης}
τίνα; 

\speaker{Εὐθύφρων}
\versification{[4a]}
ὃν
διώκων
αὖ\footnote{"again", pointing back to his earlier negative court experiences.}
δοκῶ\footnote{The people of the court consider Euthyphro μαίνεσθαι.}
μαίνεσθαι.\footnote{In this case "crazy", not "angry".}

\speaker{Σωκράτης}
τί
δέ;
πετόμενόν
τινα
διώκεις;\footnote{proverbial "go on a goose chase"}

\speaker{Εὐθύφρων}
πολλοῦ\footnote{modifying δεῖ, saying that much would be necessary for the person to πέτεσθαι, given his age.}
γε
δεῖ
πέτεσθαι,
ὅς
γε
τυγχάνει
ὢν
εὖ
μάλα\footnote{μάλα - very}
πρεσβύτης.

\speaker{Σωκράτης}
τίς
οὗτος;

\speaker{Εὐθύφρων}
ὁ
ἐμὸς
πατήρ.

\speaker{Σωκράτης}
ὁ
σός,
ὦ
βέλτιστε;\footnote{βέλτιστος - dear friend}

\speaker{Εὐθύφρων}
πάνυ\footnote{πάνυ - altogether, entirely}
μὲν
οὖν.

\speaker{Σωκράτης}
ἔστιν
δὲ
τί
τὸ
ἔγκλημα\footnote{ἔγκλημα - accusation, charge}
καὶ
τίνος
ἡ
δίκη;

\speaker{Εὐθύφρων}
φόνου,
ὦ
Σώκρατες.

\speaker{Σωκράτης}
Ἡράκλεις.
ἦ\footnote{ἦ - truly}
που,\footnote{πού - anywhere, somewhere}
ὦ
Εὐθύφρων,
ἀγνοεῖται
ὑπὸ
τῶν
πολλῶν
ὅπῃ\footnote{ὅπη - by which. ὅπῃ...ἔχει as a whole is the subject of ἀγνοεῖται.}
ποτὲ
ὀρθῶς
ἔχει·\footnote{Burnet mentions that emendations are suggested for this text (which he thinks are unnecessary), so at least others find this confusing as well.}
οὐ
γὰρ
οἶμαί
γε
τοῦ
ἐπιτυχόντος\footnote{ἐπιτυγχάνω - I obtain, attain to, reach; find, happen upon. LSJ lists this passage with the latter meaning but substantivally, "the first person happened upon" or "any chance person", with Burnet and Emlyn-Jones. In other words, most people. Perhaps contra Bailly, whose explanation I don't understand yet.}
\versification{[4b]}
[ὀρθῶς]
αὐτὸ
πρᾶξαι
ἀλλὰ
πόρρω\footnote{πόρρω + genitive = "far into", Emlyn-Jones. LSJ lists this under the idea of reaching a high point of something, in this case wisdom.}
που
ἤδη
σοφίας
ἐλαύνοντος.

\speaker{Εὐθύφρων}
πόρρω
μέντοι
νὴ\footnote{"by", seems to be used when addressing deities.}
Δία,
ὦ
Σώκρατες.

\speaker{Σωκράτης}
ἔστιν
δὲ
δὴ
τῶν
οἰκείων
τις
ὁ
τεθνεὼς
ὑπὸ
τοῦ
σοῦ
πατρός;
ἢ
δῆλα
δή;
οὐ\footnote{Comparable to the massing of particles in 2a3.}
γὰρ
ἄν
που
ὑπέρ
γε
ἀλλοτρίου\footnote{ἀλλότριος - belonging to another. In this case, referring to someone outside the household.}
ἐπεξῇσθα\footnote{ἐπέξειμι - I go out against; prosecute. The one prosecuted in the dative, the charge in the genitive.}
φόνου
αὐτῷ.

\speaker{Εὐθύφρων}
γελοῖον,
ὦ
Σώκρατες,
ὅτι
οἴει
τι
διαφέρειν
εἴτε
ἀλλότριος
εἴτε
οἰκεῖος
ὁ
τεθνεώς,
ἀλλ᾽
οὐ
τοῦτο
μόνον
δεῖν
φυλάττειν,
εἴτε
ἐν
δίκῃ\footnote{ἐν δίκῃ - lawfully}
ἔκτεινεν
ὁ
κτείνας
εἴτε
μή,
καὶ
εἰ
μὲν
ἐν
δίκῃ,
ἐᾶν,\footnote{ἐάω - I permit}
εἰ
δὲ
μή,
ἐπεξιέναι,
ἐάνπερ
ὁ
κτείνας
συνέστιός\footnote{συνέστιος - sharing one's hearth or house}
\versification{[4c]}
σοι
καὶ
ὁμοτράπεζος\footnote{ὁμοτράπεζος - sharing a table}
ᾖ·
ἴσον
γὰρ
τὸ
μίασμα\footnote{μίασμα - stain, defilement}
γίγνεται
ἐὰν
συνῇς
τῷ
τοιούτῳ
συνειδὼς
καὶ
μὴ
ἀφοσιοῖς\footnote{ἀφοσιόω - to purify from guilt}
σεαυτόν
τε
καὶ
ἐκεῖνον
τῇ
δίκῃ
ἐπεξιών.
ἐπεὶ
ὅ
γε
ἀποθανὼν
πελάτης\footnote{πελάτης - one who seeks protection, client, dependant}
τις
ἦν
ἐμός,\footnote{He was a laborer for Euthyphro}
καὶ
ὡς
ἐγεωργοῦμεν
ἐν
τῇ
Νάξῳ,\footnote{A city under the control of Athens until 404.}
ἐθήτευεν\footnote{θητεύω - I am a serf, menial laborer}
ἐκεῖ
παρ᾽
ἡμῖν.
παροινήσας\footnote{παροινέω - I play drunken tricks. Michael, we've seen this word before. Check out True Story, 107-8 and story of Diogenes.}
οὖν
καὶ
ὀργισθεὶς
τῶν
οἰκετῶν
τινι
τῶν
ἡμετέρων
ἀποσφάττει\footnote{ἀποσφάζω - I the throat of x}
αὐτόν.
ὁ
οὖν
πατὴρ
συνδήσας
τοὺς
πόδας
καὶ
τὰς
χεῖρας
αὐτοῦ,
καταβαλὼν
εἰς
τάφρον\footnote{τάφρος - ditch, tent}
τινά,
πέμπει
δεῦρο
ἄνδρα
πευσόμενον\footnote{πυνθάνομαι - I learn by hearsay, inquiry}
τοῦ
ἐξηγητοῦ\footnote{ἐξηγητής - one who leads on, adviser}
ὅτι
χρείη\footnote{\parse{pr act opt 3rd sg} of χρή - it is necessary}
\versification{[4d]}
ποιεῖν.
ἐν
δὲ
τούτῳ
τῷ
χρόνῳ
τοῦ
δεδεμένου
ὠλιγώρει\footnote{ὀλιγωρέω - I despise, hold in low esteem}
τε
καὶ
ἠμέλει\footnote{ἀμελέω - I neglect, disregard}
ὡς
ἀνδροφόνου
καὶ
οὐδὲν
ὂν\footnote{\parse{pr act ptcpl} of εἰμί}
πρᾶγμα
εἰ
καὶ
ἀποθάνοι,\footnote{\parse{aor act opt 3rd sg} of ἀποθνήσκω}
ὅπερ\footnote{ὅπερ - the very one which}
οὖν
καὶ
ἔπαθεν·
ὑπὸ
γὰρ
λιμοῦ\footnote{λιμός - hunger}
καὶ
ῥίγους\footnote{ῥῖγος - frost, cold}
καὶ
τῶν
δεσμῶν
ἀποθνῄσκει
πρὶν
τὸν
ἄγγελον
παρὰ
τοῦ
ἐξηγητοῦ
ἀφικέσθαι.
ταῦτα
δὴ
οὖν
καὶ
ἀγανακτεῖ\footnote{ἀγανακτέω - I am indignant}
ὅ
τε
πατὴρ
καὶ
οἱ
ἄλλοι
οἰκεῖοι,
ὅτι
ἐγὼ
ὑπὲρ
τοῦ
ἀνδροφόνου
τῷ
πατρὶ
φόνου
ἐπεξέρχομαι
οὔτε
ἀποκτείναντι,
ὥς
φασιν
ἐκεῖνοι,
οὔτ᾽
εἰ
ὅτι
μάλιστα
ἀπέκτεινεν,
ἀνδροφόνου
γε
ὄντος
τοῦ
ἀποθανόντος,
οὐ
δεῖν
φροντίζειν
ὑπὲρ
τοῦ
τοιούτου
--
ἀνόσιον\footnote{ἀνόσιος - unholy, profane}
\versification{[4e]}
γὰρ
εἶναι
τὸ
ὑὸν\footnote{ὑόν = υἱόν}
πατρὶ
φόνου
ἐπεξιέναι\footnote{ἐπέξειμι - I prosecute. Can take a dative or accusative direct object.}
-
κακῶς
εἰδότες,
ὦ
Σώκρατες,
τὸ
θεῖον
ὡς
ἔχει
τοῦ
ὁσίου\footnote{ὅσιος - holy}
τε
πέρι
καὶ
τοῦ
ἀνοσίου.

\speaker{Σωκράτης}
σὺ
δὲ
δὴ
πρὸς
Διός,
ὦ
Εὐθύφρων,
οὑτωσὶ\footnote{οὑτωσί - strengthened form of οὕτως}
ἀκριβῶς\footnote{ἀκριβῶς - carefully, precisely}
οἴει
ἐπίστασθαι\footnote{Probably from ἐπίσταμαι and not ἐφίστημι.}
περὶ
τῶν
θείων
ὅπῃ
ἔχει,
καὶ
τῶν
ὁσίων
τε
καὶ
ἀνοσίων,
ὥστε
τούτων
οὕτω
πραχθέντων
ὡς
σὺ
λέγεις,
οὐ
φοβῇ
δικαζόμενος
τῷ
πατρὶ
ὅπως
μὴ
αὖ\footnote{Pointing back to the deed of his father. Socrates is questioning if he is inadvertently doing wrong as he supposes his father is.}
σὺ
ἀνόσιον
πρᾶγμα
τυγχάνῃς
πράττων;

\speaker{Εὐθύφρων}
οὐδὲν
γὰρ
ἄν
μου
ὄφελος
εἴη,
ὦ
Σώκρατες,
οὐδέ
\versification{[5a]}
τῳ\footnote{This is unclear to me but probably a form of τοί (an enclytic, which explains the lack of accent), meaning something like "therefore". See Smyth section 2987. See also the short entry in LSJ on τῷ for more relevant ideas.}
ἂν
διαφέροι\footnote{\parse{pr act opt 3rd sg} of διαφέρω.}
Εὐθύφρων
τῶν
πολλῶν
ἀνθρώπων,
εἰ
μὴ
τὰ
τοιαῦτα
πάντα
ἀκριβῶς
εἰδείην.\footnote{\parse{perf act opt ,1st sg} of οἶδα.}

\speaker{Σωκράτης}
ἆρ᾽
οὖν
μοι,
ὦ
θαυμάσιε
Εὐθύφρων,
κράτιστόν
ἐστι
μαθητῇ
σῷ
γενέσθαι,\footnote{Oddly, this stative verb joins to datives, μοι and μαθητῇ σῷ. Bailly calls this complimentary but you could probably fit this in with Wallace's category of epexegetical infinitive, modifying κράτιστον.}
καὶ
πρὸ
τῆς
γραφῆς
τῆς
πρὸς
Μέλητον
αὐτὰ
ταῦτα
προκαλεῖσθαι\footnote{προκαλέω - I challenge. In Athenian law this is a technical term, referring to a πρόκλησις, a pre-trial event during which one party either asks for something or demands something from the other party. Whatever came out of this could be used in the trial (Burnet, 108; Bailly, 43)}
αὐτόν,
λέγοντα
ὅτι
ἔγωγε
καὶ
ἐν
τῷ
ἔμπροσθεν
χρόνῳ
τὰ
θεῖα
περὶ
πολλοῦ
ἐποιούμην
εἰδέναι,\footnote{ἐποιούμην εἰδέναι - combined express some notion of endeavoring to know.}
καὶ
νῦν
ἐπειδή
με\footnote{subject of the infinitive ἐξαμαρτάνειν in indirect discourse with φησι.}
ἐκεῖνος
αὐτοσχεδιάζοντά\footnote{αὐτοσχεδιάζω - I speak off-hand, extemporize; speak unadvisedly. LSJ lists the latter as the meaning here.}
φησι
καὶ
καινοτομοῦντα\footnote{καινοτομέω - I innovate}
περὶ
τῶν
θείων
ἐξαμαρτάνειν,\footnote{ἐξαμαρτάνω - I fail}
μαθητὴς
δὴ
γέγονα
σός
--
“καὶ
εἰ
μέν,
ὦ
Μέλητε”,
φαίην\footnote{\parse{pr act opt 1st sg} of φήμι}
ἄν,
“Εὐθύφρονα
ὁμολογεῖς
\versification{[5b]}
σοφὸν
εἶναι
τὰ
τοιαῦτα,
[καὶ]
ὀρθῶς
νομίζειν\footnote{Fowler applies this to Socrates, Jowett to Euthyphro. }
καὶ
ἐμὲ
ἡγοῦ\footnote{I'm not sure what this means here.}
καὶ
μὴ
δικάζου·\footnote{"do (not) prosecute me"}
εἰ
δὲ
μή,
ἐκείνῳ
τῷ
διδασκάλῳ
λάχε\footnote{\parse{aor act impv 2nd sg}.}
δίκην\footnote{λαγχάνω + δίκην means "allow to bring a suit" in Athenian law terminology. Cf. LSJ λαγχάνω I.3}
πρότερον
ἢ
ἐμοί,
ὡς
τοὺς
πρεσβυτέρους
διαφθείροντι\footnote{Dat. in agreement with the previous διδασκάλῳ.}
ἐμέ
τε
καὶ
τὸν
αὑτοῦ
πατέρα,
ἐμὲ
μὲν
διδάσκοντι,
ἐκεῖνον
δὲ
νουθετοῦντί\footnote{νουθετέω - exhort, correct; warn; advise}
τε
καὶ
κολάζοντι”\footnote{κολάζω - chastise, correct, punish; check, restrain}
--
καὶ
ἂν
μή
μοι
πείθηται\footnote{πείθω - persuade}
μηδὲ
ἀφίῃ\footnote{\parse{pres subj act 3rd sg} of ἀφίημι.}
τῆς
δίκης
ἢ
ἀντ᾽
ἐμοῦ
γράφηται
σέ,
αὐτὰ
ταῦτα
λέγειν
ἐν
τῷ
δικαστηρίῳ
ἃ
προυκαλούμην\footnote{\parse{imp act ind 1st sg} of προκαλέω.}
αὐτόν;

\speaker{Εὐθύφρων}
ναὶ
μὰ
Δία,
ὦ
Σώκρατες,
εἰ
ἄρα
ἐμὲ
ἐπιχειρήσειε\footnote{ἐπιχειρέω - I attempt}
\versification{[5c]}
γράφεσθαι,
εὕροιμ᾽\footnote{\parse{aor opt act 1st sg} of εὑρίσκω}
ἄν,
ὡς
οἶμαι,
ὅπῃ
σαθρός\footnote{σαθρός - unsound, weak}
ἐστιν,
καὶ
πολὺ
ἂν
ἡμῖν
πρότερον
περὶ
ἐκείνου
λόγος\footnote{Here, "case".}
ἐγένετο
ἐν
τῷ
δικαστηρίῳ
ἢ
περὶ
ἐμοῦ.

\speaker{Σωκράτης}
καὶ
ἐγώ
τοι,\footnote{Αnother instance of the enclitic τοί, as we saw in 5a, meaning "surely, doubtless". Smyth section 2984 is helpful.}
ὦ
φίλε
ἑταῖρε,
ταῦτα
γιγνώσκων
μαθητὴς
ἐπιθυμῶ
γενέσθαι
σός,
εἰδὼς
ὅτι
καὶ
ἄλλος
πού
τις
καὶ
ὁ
Μέλητος
οὗτος
σὲ
μὲν
οὐδὲ
δοκεῖ\footnote{Bailly and Burnet (similarly Emlyn-Jones) takes this as "pretend" (cf. LSJ δοκέω I.4), implying avoidance because Euthyphro would be a fierce opponent. Jowett and Fowler take it as "appears/seems" (cf. LSJ δοκέω II), implying that no one seems to take much account of Euthyphro.}
ὁρᾶν,
ἐμὲ
δὲ
οὕτως
ὀξέως
[ἀτεχνῶς]
καὶ
ῥᾳδίως
κατεῖδεν\footnote{κατεῖδον - look down, observe}
ὥστε
ἀσεβείας
ἐγράψατο.
νῦν
οὖν
πρὸς
Διὸς
λέγε
μοι
ὃ
νυνδὴ
σαφῶς
εἰδέναι
διισχυρίζου,\footnote{\parse{pres mid imperative 2nd sg} of διισχυρίζομαι - I affirm confidently}
ποῖόν
τι
τὸ
εὐσεβὲς
φῂς
εἶναι
καὶ
τὸ
ἀσεβὲς
\versification{[5d]}
καὶ
περὶ
φόνου
καὶ
περὶ
τῶν
ἄλλων;\footnote{Socrates is asking for a general definition, not just something that would apply to murder.}
ἢ
οὐ
ταὐτόν
ἐστιν
ἐν
πάσῃ
πράξει
τὸ
ὅσιον
αὐτὸ
αὑτῷ,
καὶ
τὸ
ἀνόσιον
αὖ
τοῦ
μὲν
ὁσίου
παντὸς
ἐναντίον,\footnote{τοῦ μὲν ὁσίου παντὸς ἐναντίον, "opposite of every holy thing"}
αὐτὸ
δὲ
αὑτῷ
ὅμοιον
καὶ
ἔχον
μίαν
τινὰ
ἰδέαν\footnote{ἔχον μίαν τινὰ ἰδέαν - "having a single/core idea"}
κατὰ
τὴν
ἀνοσιότητα
πᾶν
ὅτιπερ
ἂν
μέλλῃ
ἀνόσιον
εἶναι;

\speaker{Εὐθύφρων}
πάντως
δήπου,\footnote{δήπου - of course, we know}
ὦ
Σώκρατες.

\speaker{Σωκράτης}
λέγε
δή,
τί
φῂς
εἶναι
τὸ
ὅσιον
καὶ
τί
τὸ
ἀνόσιον;

\speaker{Εὐθύφρων}
λέγω
τοίνυν
ὅτι
τὸ
μὲν
ὅσιόν
ἐστιν
ὅπερ\footnote{ὅπερ - the very one which}
ἐγὼ
νῦν
ποιῶ,
τῷ
ἀδικοῦντι
ἢ
περὶ
φόνους
ἢ
περὶ
ἱερῶν\footnote{here rendered "consecrated property" by Burnet}
κλοπὰς\footnote{κλοπή - theft}
ἤ
τι
ἄλλο
τῶν
τοιούτων
ἐξαμαρτάνοντι
ἐπεξιέναι,\footnote{ἐπίξειμι - I go out against, or in a legal context, prosecute. Everything from τῷ ἀδικοῦντι to here is governed by this infinitive.}
ἐάντε
πατὴρ
\versification{[5e]}
ὢν
τυγχάνῃ\footnote{τυγχάνω + participle of εἰμί has occurred several times so far in this book. Often the two together expresses either coincidence or is essentially quivalent to just εἰμί (LSJ A.II). Here it is the former.}
ἐάντε
μήτηρ
ἐάντε
ἄλλος
ὁστισοῦν\footnote{ὁστισοῦν = nom. sg. of ὅστις},
τὸ
δὲ
μὴ
ἐπεξιέναι
ἀνόσιον·\footnote{This last clause summarizes the previous point: "not prosecuting (him is) unholy."}
ἐπεί,
ὦ
Σώκρατες,
θέασαι\footnote{Impv.}
ὡς
μέγα
σοι
ἐρῶ
τεκμήριον\footnote{τεκμήριον - a sign, certain proof}
τοῦ
νόμου
ὅτι
οὕτως
ἔχει
--
ὃ\footnote{Referring to τεκμήριον.}
καὶ
ἄλλοις
ἤδη
εἶπον,
ὅτι
ταῦτα
ὀρθῶς
ἂν
εἴη\footnote{\parse{pres opt act 3rd sg} of εἰμί}
οὕτω
γιγνόμενα
--
μὴ
ἐπιτρέπειν\footnote{ἐπιτρέπω - here "give up, yield, give way" (cf. LSJ ἐπιτρέπω II).}
τῷ
ἀσεβοῦντι\footnote{The idea of "to go unprosecuted" is implied here, from the μὴ ἐπεξιέναι just before.}
μηδ᾽
ἂν
ὁστισοῦν
τυγχάνῃ
ὤν.
αὐτοὶ\footnote{What follows in the τεκμήριον mentioned above.}
γὰρ
οἱ
ἄνθρωποι
τυγχάνουσι\footnote{Expressing coincidence again, "coincidentally they already think that...".}
νομίζοντες
τὸν
Δία
τῶν
θεῶν
ἄριστον
καὶ
δικαιότατον,
\versification{[6a]}
καὶ
τοῦτον
ὁμολογοῦσι
τὸν
αὑτοῦ
πατέρα\footnote{Χρόνος}
δῆσαι\footnote{Here referring to Zeus' activities against his father. See Hesiod, Theogony, Theogony, 453-492.}
ὅτι
τοὺς
ὑεῖς\footnote{ὑεῖς = υἱούς}
κατέπινεν\footnote{καταπίνω - swallow}
οὐκ
ἐν
δίκῃ,\footnote{ἐν δίκῃ is adverbially, so "not rightly". Similarly, see usage in 4b.}
κἀκεῖνόν\footnote{Referring back to πατέρα, not Zeus, because it was Chronos who castrated his father. So Euthyphro is working backwards in time. Hesiod, Theogony, 176-206}
γε
αὖ
τὸν
αὑτοῦ
πατέρα\footnote{Οὐρανός}
ἐκτεμεῖν
δι᾽
ἕτερα
τοιαῦτα·
ἐμοὶ
δὲ
χαλεπαίνουσιν\footnote{χαλεπαίνω - I am severe, sore, grievous}
ὅτι
τῷ
πατρὶ
ἐπεξέρχομαι
ἀδικοῦντι,
καὶ
οὕτως
αὐτοὶ
αὑτοῖς
τὰ
ἐναντία\footnote{ἐναντία - opposite, contrary}
λέγουσι
περί
τε
τῶν
θεῶν
καὶ
περὶ
ἐμοῦ.
