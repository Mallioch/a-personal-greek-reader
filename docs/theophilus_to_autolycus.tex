\sect{Theophilus to Autolycus}

\subsect{Book 1}

\versification{1.1}
Στωμύλον μὲν οὖν στόμα καὶ φράσις εὐεπὴς τέρψιν παρέχει καὶ ἔπαινον πρὸς κενὴν δόξαν ἀθλίοις ἀνθρώποις ἔχουσι τὸν νοῦν κατεφθαρμένον· ὁ δὲ τῆς ἀληθείας ἐραστὴς οὐ προσέχει λόγοις μεμιαμμένοις, ἀλλὰ ἐξετάζει τὸ ἔργον τοῦ λόγου τί καὶ ὁποῖόν ἐστιν. Ἐπειδὴ οὖν, ὦ ἑταῖρε, κατέπληξάς με λόγοις κενοῖς καυχησάμενος ἐν τοῖς θεοῖς σου τοῖς λιθίνοις καὶ ξυλίνοις, ἐλατοῖς τε καὶ χωνευτοῖς καὶ πλαστοῖς καὶ γραπτοῖς, οἳ οὔτε βλέπουσιν οὔτε ἀκούουσιν (εἰσὶ γὰρ εἴδωλα καὶ ἔργα χειρῶν ἀνθρώπων), ἔτι δὲ φῄς με καὶ Χριστιανὸν ὡς κακὸν τοὔνομα φοροῦντα, ἐγὼ μὲν οὖν ὁμολογῶ εἶναι Χριστιανός, καὶ φορῶ τὸ θεοφιλὲς ὄνομα τοῦτο ἐλπίζων εὔχρηστος εἶναι τῷ θεῷ. Οὐ γὰρ ὡς σὺ ὑπολαμβάνεις, χαλεπὸν εἶναι τοὔνομα τοῦ θεοῦ, οὕτως ἔχει: ἴσως δὲ ἔτι αὐτὸς σὺ ἄχρηστος ὢν τῷ θεῷ περὶ τοῦ θεοῦ οὕτως φρονεῖς.

\versification{1.2}
Ἀλλὰ καὶ ἐὰν φῇς <<Δεῖξόν μοι τὸν θεόν σου>>, κἀγώ σοι εἴποιμι ἄν <<Δεῖξόν μοι τὸν ἄνθρωπόν σου κἀγώ σοι δείξω τὸν θεόν μου>>. Ἐπεὶ δεῖξον βλέποντας τοὺς ὀφθαλμοὺς τῆς ψυχῆς σου, καὶ τὰ ὦτα τῆς καρδίας σου ἀκούοντα. Ὥσπερ γὰρ οἱ βλέποντες τοῖς ὀφθαλμοῖς τοῦ σώματος κατανοοῦσι τὴν τοῦ βίου καὶ ἐπίγειον πραγματείαν, ἅμα δοκιμάζοντες τὰ διαφέροντα, ἤτοι φῶς ἢ σκότος, ἢ λευκὸν ἢ μέλαν, ἢ ἀειδὲς ἢ εὔμορφον, ἢ εὔρυθμον καὶ εὔμετρον ἢ ἄρυθμον καὶ ἄμετρον, ἢ ὑπέρμετρον ἢ κόλουρον, ὁμοίως δὲ καὶ τὰ ὑπ᾽ ἀκοὴν πίπτοντα, ἢ ὀξύφωνα ἢ βαρύφωνα ἢ ἡδύφωνα, οὕτως ἔχοι ἂν καὶ περὶ τὰ ὦτα τῆς καρδίας καὶ τοὺς ὀφθαλμοὺς τοὺς τῆς ψυχῆς δύνασθαι θεὸν θεάσασθαι. βλέπεται γὰρ θεὸς τοῖς δυναμένοις αὐτὸν ὁρᾶν, ἐπὰν ἔχωσι τοὺς ὀφθαλμοὺς ἀνεῳγμένους τῆς ψυχῆς. Πάντες μὲν γὰρ ἔχουσι τοὺς ὀφθαλμούς, ἀλλὰ ἔνιοι ὑποκεχυμένους καὶ μὴ βλέποντας τὸ φῶς τοῦ ἡλίου. Καὶ οὐ παρὰ τὸ μὴ βλέπειν τοὺς τυφλοὺς ἤδη καὶ οὐκ ἔστιν τὸ φῶς τοῦ ἡλίου φαῖνον, ἀλλὰ ἑαυτοὺς αἰτιάσθωσαν οἱ τυφλοὶ καὶ τοὺς ἑαυτῶν ὀφθαλμούς. Οὕτως καὶ σύ, ὦ ἄνθρωπε, ἔχεις ὑποκεχυμένους τοὺς ὀφθαλμοὺς τῆς ψυχῆς σου ὑπὸ τῶν ἁμαρτημάτων καὶ τῶν πράξεών σου τῶν πονηρῶν. Ὥσπερ ἔσοπτρον ἐστιλβωμένον, οὕτως δεῖ τὸν ἄνθρωπον ἔχειν καθαρὰν ψυχήν. Ἐπὰν οὖν ᾖ ἰὸς ἐν τῷ ἐσόπτρῳ, οὐ δύναται ὁρᾶσθαι τὸ πρόσωπον τοῦ ἀνθρώπου ἐν τῷ ἐσόπτρῳ:  οὕτως καὶ ὅταν ᾖ ἁμαρτία ἐν τῷ ἀνθρώπῳ, οὐ δύναται ὁ τοιοῦτος ἄνθρωπος θεωρεῖν τὸν θεόν. Δεῖξον οὖν καὶ σὺ σεαυτόν, εἰ οὐκ εἶ μοιχός, εἰ οὐκ εἶ πόρνος, εἰ οὐκ εἶ κλέπτης, εἰ οὐκ εἶ ἅρπαξ, εἰ οὐκ εἶ ἀποστερητής, εἰ οὐκ εἶ ἀρσενοκοίτης, εἰ οὐκ εἶ ὑβριστής, εἰ οὐκ εἶ λοίδορος, εἰ οὐκ ὀργίλος, εἰ οὐ φθονερός, εἰ οὐκ ἀλαζών, εἰ οὐχ ὑπερόπτης, εἰ οὐ πλήκτης, εἰ οὐ φιλάργυρος, εἰ οὐ γονεῦσιν ἀπειθής, εἰ οὐ τὰ τέκνα σου πωλεῖς. Τοῖς γὰρ ταῦτα πράσσουσιν ὁ θεὸς οὐκ ἐμφανίζεται, ἐὰν μὴ πρῶτον ἑαυτοὺς καθαρίσωσιν ἀπὸ παντὸς μολυσμοῦ. Καὶ σοὶ οὖν ἅπαντα ἐπισκοτεῖ, καθάπερ ὕλης ἐπιφορὰ ἐπὰν γένηται τοῖς ὀφθαλμοῖς πρὸς τὸ μὴ δύνασθαι ἀτενίσαι τὸ φῶς τοῦ ἡλίου: οὕτως καὶ σοί, ὦ ἄνθρωπε, ἐπισκοτοῦσιν αἱ ἀσέβειαι πρὸς τὸ μὴ δύνασθαί σε ὁρᾶν τὸν θεόν.